\documentclass[12pt,a4paper]{article}
\usepackage[margin=1in]{geometry}

\usepackage{setspace,mdwlist,comment}
\setlength{\marginparwidth}{.5in}
\setlength{\parindent}{0in}
\setlength{\parskip}{1em}

% mini table of contents
\usepackage{minitoc}
\dosecttoc % make section toc
\setcounter{secttocdepth}{2} % subsection depth
\renewcommand{\stctitle}{} % no title
\nostcpagenumbers

\usepackage{natbib}
\usepackage{bibentry}
\newcommand{\lecture}[3][\DefaultOpt]{%
  \def\DefaultOpt{#2}%
  \subsection[#1]{#2}#3\\
}
\newcommand{\reading}[2][]{\noindent -- {#1}\bibentry{#2}.\vspace{.25em}\\}
\newcommand{\textbook}[2][]{\noindent -- {#1} from #2.\vspace{.25em}\\}
\newcommand{\thomas}{\vspace{1em}\noindent Instructor: Thomas\\}
\newcommand{\seealso}{\noindent \emph{See also these recommended readings:}\\}
\newcommand{\topic}[1]{\noindent \textbf{#1}\\}

\usepackage{booktabs}

\usepackage{hyperref}
\hypersetup{
    bookmarks=true,         % show bookmarks bar?
    unicode=false,          % non-Latin characters in Acrobat’s bookmarks
    pdftoolbar=true,        % show Acrobat’s toolbar?
    pdfmenubar=true,        % show Acrobat’s menu?
    pdffitwindow=false,     % window fit to page when opened
    pdfstartview={FitH},    % fits the width of the page to the window
    pdftitle={Syllabus: Research Design in Political Science},    % title
    pdfauthor={Thomas J. Leeper},     % author
    pdfsubject={Government} {Political Science},   % subject of the document
    pdfnewwindow=true,      % links in new window
    pdfborder={0 0 0}
}

\title{\textbf{Research Design in Political Science}\\Department of Government, LSE\\2015-2016}
\author{}
\date{}

\begin{document}
\nobibliography*
\faketableofcontents

\maketitle

\vspace{-4em}

\begin{minipage}[b]{0.5\linewidth}
\textit{Instructor}\\
Thomas J. Leeper\\
Department of Government\\
Connaught House 3.21\\
\href{mailto:T.Leeper@lse.ac.uk}{T.Leeper@lse.ac.uk}\\
\end{minipage}
\begin{minipage}[b]{0.5\linewidth}
\textit{GTA}\\
Bernardo Rangoni\\
\hspace{1em}\\
\hspace{1em}\\
\hspace{1em}\\
\end{minipage}

\section{Introduction}
The course will introduce students to the fundamentals of research design in political science. The course will cover a range of topics, starting from the formulation of research topics and research questions, the development of theory and empirically testable hypotheses, the design of data collection activities, and basic qualitative and quantitative data analysis techniques. 

The course will address a variety of approaches to empirical political science research including experimental and quasi-experimental designs, large-n survey research, small-n case selection, and comparative/historical comparisons. As a result, topics covered in the course will be varied and span all areas of political science including political behaviour, institutions, comparative politics, international relations, and public administration.

Every week will involve a lecture (\textbf{Tuesday 14:00-15:00 in CLM.7.03}), followed by a class with the Graduate Teaching Assistant. 

By the end of the course, students will be able to:

\begin{enumerate*}
\item Identify interesting political science research questions and formulate theories and hypotheses that answer them
\item Describe and operationalize concepts from political science theories
\item Evaluate the strengths and weaknesses of different approaches to empirical research
\item Apply political science theories to the design of original research
\end{enumerate*}

\section{Learning Assessment and Feedback}

Students will be evaluated through (1) a 2-hour written exam covering the full breadth of course content and (2) a 3000-word written paper applying course material in the form of a research design proposal. The final mark will reflect an equal weighting of both forms of assessment.

The \textit{written exam} covers the full breadth of material from the course and will test students' knowledge of course content, including concept definition, the appraisal of political science theories, the generation of hypotheses, and --- most importantly --- the appropriateness of different research designs for answering specific research questions. This will count for 50\% of the final mark.

The \textit{research design paper} should outline the basic elements of a novel research project, namely a research question, theoretical contribution, testable hypotheses, and a description of the proposed data collection and analysis. Unlike the written exam, this paper should focus narrowly on a topic of the student's choice and display a greater depth of understanding of a smaller set of ideas raised in the course. This will count for 50\% of the final mark.

As formative work in preparation for both exam forms, students will complete short ``problem set'' assignments, approximately every other week (see course schedule for details), which allow them to apply material from the course to concrete political science examples (e.g., identifying design elements of a published research paper; proposing strategies for answering a given research question, etc.). While these formative assessments do not count toward the final mark, they provide an opportunity for peer and instructor feedback.

\subsection{Problem Sets}

The topic of each problem set and the due date for each are as follows:

\begin{center}
\begin{tabular}{ll} \toprule
Assignment & Due Date \\ \midrule
Concepts and measurement & Tuesday Oct. 27 \\
Theory and hypothesis generation & Tuesday Nov. 10 \\
Case selection & Tuesday Nov. 24\\
Text analysis & Tuesday Dec. 15 \\
Interviewing & Tuesday Jan. 26\\
Basic Statistics & Tuesday Feb. 23\\
Regression analysis & Tuesday Mar. 8\\
Experimentation & Tuesday Mar. 22\\ \bottomrule
\end{tabular}
\end{center}


\section{Reading Material}

The following texts are \textbf{required} for the course:

\begin{enumerate}
\item \reading{Gerring2012a}
\item \reading{KellstedtWhitten2013}
\end{enumerate}

We will use the entire Kellstedt and Whitten text and most of the Gerring text as core readings for the course. Additional readings for each week are listed on the course schedule. Journal articles should be available online and selections from books are available as eBooks or via the library (links noted below). All readings should be completed \textit{before} the scheduled course meeting.

\textit{Recommended readings} for each course topic, which are not required to be read, are listed on the course schedule under the heading \textit{See Also}. Students may also be interested in the following general texts on research design in the social sciences:

\reading{KingKeohaneVerba1994}
\reading{ShadishCookCampbell2001}
\reading{MorganWinship2015}
\reading{AngristPischke2008}
\reading{Rosenbaum2009}
\reading{GerberGreen2012}
\reading{ImbensRubin2015}


\section{Course Website}

All material relevant to the course will be uploaded to the course Moodle site, which can be found at: \url{https://moodle.lse.ac.uk/course/view.php?id=4889}


\clearpage
\section{Schedule}
The general schedule for the course is as follows. Details on topics covered and the readings for each week are provided on the following pages. Sessions 1-10 meet during Michaelmas Term and Sessions 11-20 meet during Lent Term.\\

\secttoc

Note that there will be no lecture or class during Lent Term reading week (Feb. 15--19).

\clearpage



% MICHAELMAS TERM

% 1
\lecture{Introduction (Sep. 28)}{An overview of the course and an introduction to political science research.}

\textbook[Ch. 2]{Gerring}



% 2
\lecture{Causality: Explanation versus Prediction (Oct. 6)}{Political science is generally concerned with questions of causality. To do that we need to learn to think counterfactually. How do we know that something causes something else? How do we separate ``correlation'' from ``causation''?}

\textbook[Ch. 3]{Kellstedt and Whitten}
\reading{Holland1986}

\seealso
\textbook[Ch. 8]{Gerring}
\reading[Brady, Henry. ``Causation and Explanation in the Social Sciences'' (Ch. 8; pp.217--270) from ]{BoxSteffensmeieretal2008} % http://dx.doi.org/10.1093/oxfordhb/9780199286546.001.0001




% 3
\lecture{Concepts: ``I'll know it when I see it'' (Oct. 13)}{Before we can study something we need to know what that ``something'' is. This is concept definition. How do we define concepts and how do we separate different concepts from one another?}

\textbook[Ch. 5--6]{Gerring}
\reading{Mansbridge2003} % Rethinking Representation
\reading[pp.1--16 (Ch. 1) from ]{Dahl1971} % Polyarchy

\seealso
\reading{Goertz2005}
\reading[Appendix on Etymology (pp.241--252) from ]{Pitkin1967} % Pitkin, Hannah. 1967. 'Appendix on Etymology.' In Pitkin, The Concept of Representations, pp. 241-252.
\reading{Sartori1970}


% 4
\lecture{Measurement: Concepts in Practice (Oct. 20)}{To study something, we need to be able to observe and measure it. How do we \textit{operationalize} concepts so that we can study political phenomena? What are challenges of measuring concepts? How do we assign quantitative values to observations?}

\textbook[Ch. 5 (through p.109)]{Kellstedt and Whitten}
\textbook[Ch. 7]{Gerring}

\seealso
\reading{AdcockCollier2001} %% Adcock, R. & Collier, D. 2001. “.” American Political Science Review 95: 529-546.
\reading{Paxton2000}
\reading{MunckVerkuilen2002}


% 5
\lecture{Building Theories from Observations and Principles (Oct. 27)}{How do we create social science theories based on past evidence and novel observation? What roles do induction and deduction play in contemporary political science?}

\textbook[Ch. 1--2]{Kellstedt and Whitten}
\textbook[Ch. 3]{Gerring}




% 6
\lecture{Deriving Hypotheses from Theory (Nov. 3)}{Hypotheses are the observable implications of theories. How do we derive hypotheses from theories? How do we overcome ``observational equivalence'' wherein multiple theories yield similar expectations about the world? What does it mean to test a hypothesis?} % something on research synthesis and lit reviewing?

\reading{Fearon1991}
\reading{Tannenwald1999} % Nuclear Taboo


\seealso
\reading{Fiorina1975}
%% Albert O. Hirschman, “The Search for Paradigms as a Hindrance to Understanding.” World Politics 22 (April 1970): 329-343.
%% Bueno de Mesquita, Bruce (1985) – “Toward a scientific understanding of international conflict: A personal view,” International Studies Quarterly, 29(2), 121-136.


% 7
\lecture{Case Studies (Nov. 10)}{Case studies are in-depth examinations of a single manifestation of a political phenomenon and are one of the most common methods of inquiry in political science. What can we do with case studies? How do they help us to understand politics?}

\reading{Gerring2004} % What Is A Case Study and What Is It Good For?
\textbook[Ch. 12 (only up to p.342)]{Gerring}

\seealso
\reading{MahoneyGoertz2004}
\reading{HardingFoxMehta2002}


% 8
\lecture{Case Comparisons over Time and Place (Nov. 17)}{How do comparisons between cases help us to make inferences about causality? How do we select cases so that comparisons between them are informative about theories and hypotheses?}

\reading{DrezeSen1989} %  https://shibboleth2sp.sams.oup.com/Shibboleth.sso/Login?entityID=https://lse.ac.uk/idp&target=https://shibboleth2sp.sams.oup.com/shib?dest=http://www.oxfordscholarship.com/SHIBBOLETH?dest=http://dx.doi.org/10.1093/0198283652.001.0001
\reading{DonerRitchieSlater2005}

\seealso
\reading{Mahoney2000b}
\reading{Lijphart1971}
\reading{Mahoney1999} 
\reading{Geddes1991}
%\reading{Posner2004} %% Posner, Daniel. 2004. “The Political Salience of Cultural Difference: Why Chewas and Tumbukas are Allies in Zambia and Adversaries in Malawi.” American Political Science Review 98:4 (November) 529-46.



% 9
\lecture{Causal Mechanisms and Process-Tracing (Nov. 24)}{Aside from knowing that one thing (X) caused another thing (Y), we often want to know how that causal process worked. This is the study of ``causal mechanisms''. How do we study causal mechanisms to gain a deeper understanding of causal relationships in politics? How do we study the process by which a causal effect plays out?}


\reading[Ch. 10 (pp.325--353) from ]{MorganWinship2015} % EBOOK AVAILABLE
\reading{Brady2004} %(\url{https://www.dawsonera.com/guard/protected/dawson.jsp?name=https://lse.ac.uk/idp&dest=http://www.dawsonera.com/depp/reader/protected/external/AbstractView/S9781442203457})

\seealso
\reading{RueschemeyerStephens1997}

% 10
\lecture{Translating Texts into Interpretations and Numbers (Dec. 1)}{Primary and secondary source documents provide a written record of politically relevant events and processes. Texts can be used in a number of ways in political science research. How do we draw meaning from texts in qualitative and quantitative ways? How does textual information become useful data for making political inferences?}


\reading{Finnegan1996}
\reading{Lustick1996}
\reading{YoungSoroka2012}
% quantitative content analysis

\seealso
\reading{GrimmerStewart2013}






% LENT TERM

% 11
\lecture{Interviewing, Structured and Unstructured (Jan. 12)}{Interviewing is an integral part of political science research. Whether it is interviewing elite political actors as part of a case study or survey interviewing as part of an election poll, interviews are often how concepts are operationalized and insights obtained. How do we conduct interviews? What roles do different kinds of interviews play in political science research?}

\reading{SchaefferPresser2003}
\reading{Peabodyetal1990} 

\seealso
\reading[pp. 217-257 (Ch. 7 especially section 7.3 to end) from ]{Grovesetal2009} %  https://www.dawsonera.com/guard/protected/dawson.jsp?name=https://lse.ac.uk/idp&dest=http://www.dawsonera.com/depp/reader/protected/external/AbstractView/S9781118627327
\reading{RubinRubin2005}
\reading{Leech2002}


% 12
\lecture{Actually Talking to People: Participant Observation (Jan. 19)}{Collecting political science data often requires actually talking to human beings about their knowledge, thoughts, feelings, opinions, and actions. Some of this is done one-on-one (as we talked about in the previous week), but much of these conversations also take place in group settings. How do we talk to people in groups in a way that helps us draw inferences about politics?}

\reading{Fenno1977} %% U.S. House Members in Their Constituencies: An Exploration.

\seealso
\reading{Chong1993}
\reading{Wedeen2010}
\reading{ConoverCreweSearing1991}


% 13
\lecture{Tabulation and Visualization (Jan. 26)}{How do we summarize our observations using tables and graphs? How do we communicate our research to technical and non-technical audiences in clear and meaningful ways?}

\textbook[Ch. 5 (pp.109 to end)]{Kellstedt and Whitten}
\reading{KastellecLeoni2007}

\seealso
\reading[Especially Ch. 5 (``Chartjunk'') from ]{Tufte1983} % Tufte, Edward. 2001. 'Chartjunk.' In Tufte, The Visual Display of Quantitative Information. 2nd Edition. Graphics Press. (Ch.5)
%\reading{Babbie} % Babbie, Earl. 2015. 'The Logic of Multivariate Analysis.' in Babbie, The Practice of Social Research, 15th edition, pp. 432-449 (Ch. 15).



% 14
\lecture{Sampling and Representativeness (Feb. 2)}{In quantitative political science research, sampling is the basis of both claims about ``representativeness'' (i.e., the extent to which findings from a study apply to some well-defined population) and statistical inference (i.e., claims about whether some observation is ``statistically significant''). What is sampling? How do we sample from populations? How does sampling allow us to make inferences about populations?}

\reading[``External Validity,'' pp.83--95 from ]{ShadishCookCampbell2001}
\textbook[Ch. 6]{Kellstedt and Whitten} % probability

\seealso 
\reading[``Increasing the number of observations'' (pp. 208--230) from ]{KingKeohaneVerba1994} % https://www.dawsonera.com/guard/protected/dawson.jsp?name=https://lse.ac.uk/idp&dest=http://www.dawsonera.com/depp/reader/protected/external/AbstractView/S9781400821211


% 15
\lecture{Statistical Inference (Feb. 9)}{Random sampling allows us to quantitatively test hypotheses about empirical regularities. This allows us to make claims about ``statistical significance'' (such as whether two groups differ from one another or whether a feature of a group differs from an expectation dictated by theory). How do we use statistical significance testing in political science? How do we interpret statistical significance tests?}


\textbook[Ch. 7]{Kellstedt and Whitten} % sampling distributions, t-test, prop-test

\seealso
\reading{Robinson1950} %% unit of measurement


\subsection*{Reading Week -- No Lecture or Class (Feb. 15--19)}
\vspace{1em}


% 16
\lecture{Getting to Regression: The Workhorse of Quantitative Political Analysis (Feb. 23)}{By far the most commonly used method of quantitative analysis in political science is ``regression.'' What is regression? How do we use it? How do we interpret the results of regression analyses?} 

\textbook[Ch.8]{Kellstedt and Whitten}
-- ``Babel or babble? The evolution of language.'' \textit{The Economist}, April 14, 2011. %\url{http://www.economist.com/node/18557572}


% 17
\lecture{Matching and Regression: Accounting for Rival Explanations (Mar. 1)}{How do we use regression analysis to make causal inferences? How do we account for the fact that an outcome we are interested in might be caused by multiple events, features, or attributes of cases?}

\textbook[Ch.9--10]{Kellstedt and Whitten}
\reading{CusackIversenSoskice2007}


% 18
\lecture{Experimental Design and the Search for Quasi-Experiments (Mar. 8)}{The clearest path to causal inference is through experimentation. How does experimentation differ from observational research? Why does experimentation provide a uniquely powerful design for making causal inferences? In lieu of experimentation, how can design research around real-world variation that has quasi-experimental properties?}

\reading{GerberGreen2008}
\reading{Bhavnani2009}
\reading{CampbellRoss1968} % Connecticut Crackdown on Speeding

%\seealso 
%


% 19
\lecture{Ethics and Research Integrity (Mar. 15)}{The practice of political science research evokes numerous ethical considerations. By observing the world, political scientists potentially obtain data that is confidential or private. By intervening in the world, political scientists potentially affect real-world politics in expected and unexpected ways. How do we think about and address these and other ethical challenges of conducting research?}

\reading{BelmontReport} % Belmont Report. http://www.hhs.gov/ohrp/humansubjects/guidance/belmont.html
\reading{Simitis1994}

%% Nuremberg Code (http://www.hhs.gov/ohrp/archive/nurcode.html)
%% Belmont Report
%% Examples: Zimbardo, Tuskegee, privacy, informed consent, deception, IRB, field studies

% LSE Documents:
% http://www.lse.ac.uk/intranet/researchAndDevelopment/researchDivision/policyAndEthics/QuickGuide-Research-Ethics-web.pdf
% http://www.lse.ac.uk/intranet/LSEServices/policies/pdfs/school/resEthPolPro.pdf
% http://www.lse.ac.uk/intranet/LSEServices/policies/pdfs/school/ethCod.pdf


% 20
\lecture{Conclusion and Synthesis (Mar. 22)}{Where have we been? What have we learned? Where do we go from here?}

-- No required reading

%\seealso
%% Robert Dahl, "The Behavioral Approach in Political Science: Epitaph for a Monument to a Successful Protest." APSR (1961) 55:763- 72.
%% Charles Merriam, "The Present State of the Study of Politics," APSR (1921) 15:173-85.
%% Mahoney, J. & Goertz, G. 2006. "A Tale of Two Cultures: Contrasting Quantitative and Qualitative Research." Political Analysis 14: 227-249.
%% Lowi, Theodore J. 1992. “The State in Political Science: How We Become What We Study.” American Political Science Review 86, 1-7.


% load bibtext, but don't generate a bibliography
\bibliographystyle{plain}
\nobibliography{Syllabus}


\end{document}
