\documentclass[12pt,a4paper]{article}
\usepackage[margin=1in]{geometry}

\usepackage{setspace,mdwlist,comment}
\setlength{\marginparwidth}{.5in}
\setlength{\parindent}{0in}
\setlength{\parskip}{1em}

% mini table of contents
\usepackage{minitoc}
\dosecttoc % make section toc
\setcounter{secttocdepth}{2} % subsection depth
\renewcommand{\stctitle}{} % no title
\nostcpagenumbers

\usepackage{natbib}
\usepackage{bibentry}
\newcommand{\lecture}[3][\DefaultOpt]{%
  \def\DefaultOpt{#2}%
  \subsection[#1]{#2}#3\\
}
\newcommand{\reading}[2][]{\noindent -- {#1}\bibentry{#2}.\vspace{.25em}\\}
\newcommand{\textbook}[2][]{\noindent -- {#1} from #2.\vspace{.25em}\\}
\newcommand{\thomas}{\vspace{1em}\noindent Instructor: Thomas\\}
\newcommand{\seealso}{\noindent \emph{See also these recommended readings:}\\}
\newcommand{\topic}[1]{\noindent \textbf{#1}\\}

\usepackage{booktabs}

\usepackage{hyperref}
\hypersetup{
    bookmarks=true,         % show bookmarks bar?
    unicode=false,          % non-Latin characters in Acrobat’s bookmarks
    pdftoolbar=true,        % show Acrobat’s toolbar?
    pdfmenubar=true,        % show Acrobat’s menu?
    pdffitwindow=false,     % window fit to page when opened
    pdfstartview={FitH},    % fits the width of the page to the window
    pdftitle={Syllabus: Research Design in Political Science},    % title
    pdfauthor={Thomas J. Leeper},     % author
    pdfsubject={Government} {Political Science},   % subject of the document
    pdfnewwindow=true,      % links in new window
    pdfborder={0 0 0}
}

\title{\textbf{GV249: Research Design in Political Science}\\Department of Government\\London School of Economics and Political Science\\2016--17}
\author{}
\date{}

\begin{document}
\nobibliography*
\faketableofcontents

\maketitle

\vspace{-4em}

\begin{minipage}[t]{0.5\linewidth}
\textit{Instructor}\\
Thomas J. Leeper\\
Government Department\\
Connaught House 4.11\\
\href{mailto:t.leeper@lse.ac.uk}{T.Leeper@lse.ac.uk}\\
Office Hours: Mon 10:30--11:30\\
\phantom{Office Hours:} Fri 9:30--10:30\\
\end{minipage}
\begin{minipage}[t]{0.5\linewidth}
\textit{GTA}\\
Elena Pupaza\\
\href{mailto:E.C.Pupaza@lse.ac.uk}{E.C.Pupaza@lse.ac.uk}\\
\hspace{1em}\\
\hspace{1em}\\
\end{minipage}

\section{Introduction}
The course will introduce students to the fundamentals of research design in political science. The course will cover a range of topics, starting from the formulation of research topics and research questions, the development of theory and empirically testable hypotheses, the design of data collection activities, and basic qualitative and quantitative data analysis techniques. 

The course will address a variety of approaches to empirical political science research including experimental and quasi-experimental designs, large-n survey research, small-n case selection, and comparative/historical comparisons. As a result, topics covered in the course will be varied and span all areas of political science including political behaviour, institutions, comparative politics, international relations, and public administration.

Every week will involve the following in-person sessions:

	\begin{center}
	Lecture: \textbf{Friday 11:00--12:00 in 32L.B.09 (basement level)}\\
	Class: \textbf{Friday 13:00--14:00 in OLD.1.27}\\
	\hspace{2.8em} \textbf{Friday 15:00--16:00 in CLM 1.02}
	\end{center}

Class sessions are hosted by the Graduate Teaching Assistant.

\section{Learning Assessment and Feedback}

\subsection{Intended Learning Outcomes}

By the end of the course, students will be able to:

\begin{enumerate*}
\item Identify theories, hypotheses, and methods used in empirical political science research.
\item Apply different methods to political science research questions.
\item Analyze data to measure concepts, make comparisons, and draw inferences.
\item Define causation and the multiple ways of reaching causal inferences.
\item Communicate political science concepts, theories, and methods in writing.
\end{enumerate*}

These outcomes are indicative of what kinds of knowledge should be demonstrated on formative and summative assessments, including the exam.


\subsection{Assessment}

Students will be evaluated through (1) a 2-hour written exam covering the full breadth of course content and (2) a 3000-word written paper applying course material in the form of a research design proposal. The final mark will reflect an equal weighting of both forms of assessment.

The \textit{written exam} covers the full breadth of material from the course and will test students' knowledge of course content, including concept definition, the appraisal of political science theories, the generation of hypotheses, and --- most importantly --- the appropriateness of different research designs for answering specific research questions. Note the exam form is slightly different from that used in the 2015--16 academic year. A sample paper will be provided that conveys the structure used in 2016-17. The exam will count for 50\% of the final mark.

The \textit{research design proposal} should outline the basic elements of a novel research project, namely a research question, theoretical contribution, testable hypotheses, and a description of the proposed data collection and analysis. Unlike the written exam, this paper should focus narrowly on a topic of the student's choice and display a greater depth of understanding of a smaller set of ideas raised in the course. This will count for 50\% of the final mark. Students should work over the full academic year on the research design proposal. A schedule of formative deadlines for the proposal are:

\begin{itemize}
\item A proposal of two possible research topics is due at the end of November 2016 and will be presented in class.
\item A literature review of relevant existing research is due at the end of February 2017.
\item The final essay is due on Tuesday, 21 March 2017 at 5:00pm.
\end{itemize}

As formative work in preparation for both exam forms, students will complete short ``problem set'' assignments, approximately every other week (see course schedule for details), which allow them to apply material from the course to concrete political science examples (e.g., identifying design elements of a published research paper; proposing strategies for answering a given research question, etc.). While these formative assessments do not count toward the final mark, they provide an opportunity for peer and instructor feedback.

\subsection{Assignments}

The topic of each assignment (problem set or otherwise) and the due date for each are as follows:

\begin{center}
\begin{tabular}{lll} \toprule
\textbf{Assignment} & \textbf{Type} & \textbf{Due Date} \\ \midrule
Identifying and Evaluating Claims & Problem Set 1 & Tuesday Oct.~17 \\
Concepts and Measurement & Problem Set 2 & Tuesday Nov.~7 \\
Data Collection I & Problem Set 3 & Tuesday Nov.~21 \\
Data Collection II & Problem Set 4 & Tuesday Dec.~5 \\
Proposal Topics Presentation & Formative Presentation & Nov./Dec. \\  \midrule 
Causality & Problem Set 5 & Tuesday Jan.~16 \\
Theory Evaluation & Problem Set 6 & Tuesday Feb.~13 \\
Article Critique & Problem Set 7 & Tuesday Feb.~27 \\
Literature Review & Formative Essay & Reading Week \\ 
Statistics and Regression & Problem Set 8 & Tuesday Mar.~13 \\ \midrule 
Final Research Design Proposal & Summative Essay & Tuesday Mar.~21 \\ \midrule 
ST Exam & Summative Exam & ST TBA \\
\bottomrule
\end{tabular}
\end{center}

All assignments are due on Tuesday, via Moodle unless otherwise stated. Written and/or oral feedback will be provided by the next lecture after the end of that week. So, if an assignment is due Tuesday October 11, feedback will be provided by Friday October 21.

\subsection{Assignment Policies}

All work for the course should follow policies and procedures as described in the Government Department's Undergraduate Student Handbook\footnote{http://www.lse.ac.uk/government/StudentInformation/Current-undergraduate-students/Undergraduate-Handbook.aspx}. Students should, in particular, be aware of the LSE policy on plagiarism (\url{http://www.lse.ac.uk/socialPolicy/InformationForCurrentStudents/plagiarism.aspx}).

No late work will be accepted and given the long-term nature of the research design proposal, it is the general policy of the course not to offer extensions on this.

\subsection{Special Note for General Course Students}

Students in the General Course must complete all formative and summative coursework. In addition to the overall final mark for the course, General Course students will also receive a ``class grade'' as determined by the instructor and GTA. 

This mark will be determined by the following formula:

\begin{center}
\begin{tabular}{rl}
10\% & Class Attendance \\
10\% & Lecture and Online Participation \\
20\% & Class Participation \\
20\% & Group and Individual Projects \\
40\% & Performance on Formative Problem Sets (8 x 5\% each) \\
\end{tabular}
\end{center}

For more details on this, please see the LSE website.\footnote{\url{http://www.lse.ac.uk/study/generalCourse/currentStudentsAndAlumni/GeneralCourseAssessment.aspx}}


\section{Getting Help}

Should you encounter any difficulties during this course, there are resources available to you. The instructor and GTA hold regular office hours that you are always welcome to attend or contact us via email.

In addition to this, you may at some point find the help of the following LSE offices helpful:

\begin{itemize*}\itemsep1em

\item For issues related to the degree programme or exam policies: the Government Department undergraduate team (\href{mailto:Gov.Ug@lse.ac.uk}{gov.ug@lse.ac.uk}) or the Student Services Centre\footnote{\url{http://www.lse.ac.uk/intranet/students/supportServices/studentServicesCentre/Home.aspx}}

\item For issues of disability or health that impacts your life as a student: Disability and Wellbeing Service,\footnote{\url{http://www.lse.ac.uk/intranet/LSEServices/disabilityAndWellBeingService/home.aspx}} or for issues of mental health, stress, etc.: Student Counselling Service\footnote{\url{http://www.lse.ac.uk/intranet/students/supportServices/healthSafetyWellbeing/adviceCounselling/studentCounsellingService/Home.aspx}}

\item For academic and writing support: Teaching and Learning Centre,\footnote{\url{http://www.lse.ac.uk/intranet/LSEServices/TLC/Home.aspx}} the Language Centre,\footnote{\url{http://www.lse.ac.uk/language/EnglishProgrammes/EnglishHome.aspx}} and the Library\footnote{\url{http://www.lse.ac.uk/library/usingTheLibrary/training/Information-skills-and-resources.aspx}}

\end{itemize*}

\section{Course Materials}

\subsection{Course Website}

All material relevant to the course will be uploaded to the course Moodle site, which can be found at: \url{https://moodle.lse.ac.uk/course/view.php?id=4889}

\subsection{Textbooks and Readings}

The following text is \textbf{required} reading for the course:

\begin{itemize}
\item \reading{Toshkov2016}
\end{itemize}

We will use the entire Toshkov text as a core reading for the course. Additional readings for each week are listed on the course schedule. Journal articles should be available online and selections from books are available as eBooks or via the library's ReadingLists\@LSE service. All readings should be completed \textit{before} the scheduled lecture meeting. We may discuss readings in lecture, as well as in the subsequent class meeting.

\textit{Recommended readings} for each course topic, which are not required to be read and are not covered by the exam, are listed on the course schedule under the heading \textit{See Also}. These may be useful as further reading or in developing the research design proposal portion of the exam. Students may also be interested in the following general texts on research design in the social sciences:

\reading{KingKeohaneVerba1994}
\reading{ShadishCookCampbell2001}
\reading{Geddes2003}
\reading{Gerring2012a}
\reading{MorganWinship2015}
\reading{Seawright2016}

\noindent Students interested in gaining additional background experience with \textit{quantitative} aspects of research design and data analysis, may be interested in the following:

\reading{Imai2017}
\reading{Field2016}
\reading{KellstedtWhitten2013}
\reading{GelmanHill2006}
\reading{AngristPischke2008}
\reading{AngristPischke2015}
\reading{Monogan2016}
\reading{Rosenbaum2009}
\reading{ImbensRubin2015}
\reading{Freedman1997}

\noindent Students interested in gaining additional background experience with \textit{qualitative} aspects of research design and data analysis, may be interested in the following:

\reading{Gerring2012b}
\reading{Schreier2012}
\reading{Hay2002}


\subsection{Software}

For several weeks throughout the course, we will use the free and open source statistical analysis software R. You can download and install R on your personal computer from \url{https://cran.r-project.org/}. You may also wish to install an advanced text editor that will make it easier to use. Possibilities include RStudio (\url{https://www.rstudio.com/products/rstudio/download3/}), Notepad++ on Windows (\url{https://notepad-plus-plus.org/}), or any other listed here: \url{https://en.wikipedia.org/wiki/Comparison_of_text_editors}.



\clearpage
\section{Schedule}
The general schedule for the course is as follows. Details on topics covered and the readings for each week are provided on the following pages. Sessions 1--11 meet during Michaelmas Term and Sessions 11-20 meet during Lent Term.\\

\secttoc

Note: Lectures and classes are held in weeks 1--5,7--11 of each term. There will be no lecture or class during reading weeks (Oct.~31--Nov.~4 and Feb.~13--17).

\clearpage



\subsection*{MICHAELMAS TERM}
\vspace{1em}

% 1
\lecture{Course Introduction (Sep.~29)}{An overview of the course and an introduction to political science research. How do we identify research topics to study empirically? What makes for a good political science research questions?}

\textbook[Ch. 1 and pp.44-54]{Toshkov}
% KKV


% 2
\lecture{Concepts: ``I'll know it when I see it'' (Oct.~6)}{Before we can study something we need to know what that ``something'' is. This is concept definition. How do we define concepts and how do we separate different concepts from one another?}

\textbook[Ch. 4]{Toshkov}
\textbook[Ch. 5--6]{Gerring}
\phantom{abcde}Available from: \href{https://contentstore.cla.co.uk/secure/link?id=e3d9e19d-b22c-e611-80bd-0cc47a6bddeb}{https://contentstore.cla.co.uk/secure/link?id=e3d9e19d-b22c-e611-80bd-0cc47a6bddeb}\\
\reading{Mansbridge2003} % Rethinking Representation

\seealso
\reading{Goertz2005}
\reading[Appendix on Etymology (pp.241--252) from ]{Pitkin1967} % Pitkin, Hannah. 1967. 'Appendix on Etymology.' In Pitkin, The Concept of Representations, pp. 241-252.
\reading{Sartori1970}


% 3
\lecture{Measurement: Concepts in Practice (Oct.~13)}{To study something, we need to be able to observe and measure it. How do we \textit{operationalize} concepts so that we can study political phenomena? What are challenges of measuring concepts? How do we assign quantitative values to observations?}

\textbook[Ch. 5]{Toshkov}
\reading{AdcockCollier2001} %% Adcock, R. & Collier, D. 2001. “.” American Political Science Review 95: 529-546.
\reading{MunckVerkuilen2002}
\reading{Wickham2010}

\seealso
\reading{Prior2009a}
\textbook[Ch. 7]{Gerring}
\reading{Paxton2000}


% 4
\lecture{Tabulation and Visualization (Oct.~20)}{How do we summarize our observations using tables and graphs? How do we communicate our research to technical and non-technical audiences in clear and meaningful ways?}

\reading{Wainer1984}
\reading{KastellecLeoni2007}

\seealso
\reading[Especially Ch. 5 (``Chartjunk'') from ]{Tufte1983} % Tufte, Edward. 2001. 'Chartjunk.' In Tufte, The Visual Display of Quantitative Information. 2nd Edition. Graphics Press. (Ch.5)
\textbook[Ch. 5 (pp.109 to end)]{Kellstedt and Whitten}
%\reading{Babbie} % Babbie, Earl. 2015. 'The Logic of Multivariate Analysis.' in Babbie, The Practice of Social Research, 15th edition, pp. 432-449 (Ch. 15).
\reading{Cairo2012}
\phantom{abcde}Preview chapter available for free online from: \href{http://www.elartefuncional.com/images/Intro_chapter1.pdf}{http://www.elartefuncional.com/images/\\Intro\_chapter1.pdf}
\reading{Cairo2016}
\reading{Wainer2011}
\reading{Wickham2010}

Also, these online resources:

 -- \textit{Flowing Data}. \href{http://flowingdata.com/}{http://flowingdata.com/} \\
 -- \textit{Visual Business Intelligence}. \href{http://www.perceptualedge.com/blog/}{http://www.perceptualedge.com/blog/}\\
 -- \textit{The Functional Art}. \href{http://www.thefunctionalart.com/}{http://www.thefunctionalart.com/}\\
 -- \textit{Information is Beautiful}. \href{http://www.informationisbeautiful.net/}{http://www.informationisbeautiful.net/}\\
 -- \textit{Junk Charts}. \href{http://junkcharts.typepad.com/}{http://junkcharts.typepad.com/} \\


% 5
\lecture{Description and Evidence Gathering (Oct.~27)}{How should we gather and use \textit{descriptions} of political phenomenon, be they qualitative or quantitative in nature? What kinds of evidence can we gather to draw conclusions about the social and political world?}

\reading{Lustick1996}
% something about gathering existing quantitative data
\reading{Gerring2012b}

\seealso

\reading{Finnegan1996}


\subsection*{Reading Week -- No Lecture or Class (Oct.~30--Nov.~3)}
\vspace{1em}


% 6
\lecture{Translating Texts into Interpretations and Numbers (Nov.~10)}{Primary and secondary source documents provide a written record of politically relevant events and processes. Texts can be used in a number of ways in political science research. How do we draw meaning from texts in qualitative and quantitative ways? How does textual information become useful data for making political inferences?}


\reading{GrimmerStewart2013}
\reading{YoungSoroka2012}
% quantitative content analysis

\seealso


% 7
\lecture{Actually Talking to People (Nov.~17)}{Collecting political science data often requires actually talking to human beings about their knowledge, thoughts, feelings, opinions, and actions. Some of this is also takes place in group or public settings. How do we conduct interviews? What roles do different kinds of interviews play in political science research? How do we talk to people in natural settings in a way that helps us draw inferences about politics?}

\reading{Goffman2009} % On the Run (article)
 -- Lubet, Steven. ``Ethnography on Trial.'' \textit{The New Republic}, 15 July 2015. Available from: \href{https://newrepublic.com/article/122303/ethnography-trial}{https://newrepublic.com/article/122303/ethnography-trial}\\
\reading{SchaefferPresser2003}
\reading{Peabodyetal1990} 

\seealso

\noindent Regarding survey and interviewing methods:
\reading{SchumanPresser1981}
\reading[pp. 217-257 (Ch. 7 especially section 7.3 to end) from ]{Grovesetal2009}
\reading{Chong1993}
\reading{ConoverCreweSearing1991}
\reading{RubinRubin2005}
\reading{Leech2002}

\noindent Regarding ethnographic methods:

\reading[Especially the appendix]{Fenno1977} %% U.S. House Members in Their Constituencies: An Exploration.
\reading{Wedeen2010}
\reading[Especially Ch. 2 ]{Cramer2016}
\reading{Nielsen2012}
\reading{Mansbridge1980}


% 8
\lecture{Sampling and Representativeness (Nov.~24)}{In quantitative political science research, sampling is the basis of both claims about ``representativeness'' (i.e., the extent to which findings from a study apply to some well-defined population) and statistical inference (i.e., claims about whether some observation is ``statistically significant''). What is sampling? How do we sample from populations? How does sampling allow us to make inferences about populations?}

\reading{Hedges2013}
\reading[``External Validity,'' pp.83--95 from ]{ShadishCookCampbell2001}
\phantom{abcde}Available from: \href{https://contentstore.cla.co.uk/secure/link?id=e2d9e19d-b22c-e611-80bd-0cc47a6bddeb}{https://contentstore.cla.co.uk/secure/link?id=e2d9e19d-b22c-e611-80bd-0cc47a6bddeb} (Please note some of the terminology used in this excerpt is outdated and may read as inappropriate today.)

\seealso 
\reading[``Increasing the number of observations'' (pp. 208--230) from ]{KingKeohaneVerba1994} % https://www.dawsonera.com/guard/protected/dawson.jsp?name=https://lse.ac.uk/idp&dest=http://www.dawsonera.com/depp/reader/protected/external/AbstractView/S9781400821211
\textbook[Ch. 6]{Kellstedt and Whitten} % probability
\reading{Grovesetal2009}
\reading{Lumley2010}
\reading{Lohr2009}



% 9
\lecture{Ethics and Research Integrity (Dec.~1)}{The practice of political science research evokes numerous ethical considerations. By observing the world, political scientists potentially obtain data that is confidential or private. By intervening in the world, political scientists potentially affect real-world politics in expected and unexpected ways. How do we think about and address these and other ethical challenges of conducting research?}

\reading{BelmontReport}
 -- ``LSE Research Ethics Policy.'' Available at: \href{http://www.lse.ac.uk/intranet/LSEServices/policies/pdfs/school/resEthPolPro.pdf}{http://www.lse.ac.uk/intranet/LSEServices/\\policies/pdfs/school/resEthPolPro.pdf}


\seealso

\reading{Desposato2015}
\reading{Simitis1994}


% 10
\lecture{From Description to Causation (Dec.~8)}{When we see patterns in the social and political world, how do we know if correlations are causal? Does what comes before cause what comes after? How would we know? What limits our ability to trace the flow of causality?}

\reading{CampbellRoss1968} % Connecticut Crackdown on Speeding
\reading[Ch. 3 (pp.77--101) from ]{MorganWinship2015} % EBOOK AVAILABLE
\phantom{abcde} Available online at \href{http://ebooks.cambridge.org.gate2.library.lse.ac.uk/ebook.jsf?bid=CBO9781107587991}{Cambridge Books Online}


\subsection*{LENT TERM}
\vspace{1em}

% 11
\lecture{Causality: Developing Explanations (Jan.~12)}{Political science is generally concerned with questions of causality. To do that we need to learn to think counterfactually. How do we know that something causes something else? How do we separate ``correlation'' from ``causation''?}

\textbook[Ch. 6]{Toshkov}
\reading{Holland1986}

\seealso
\textbook[Ch. 8]{Gerring}
\reading{Shmueli2010}
\reading[Brady, Henry. ``Causation and Explanation in the Social Sciences'' (Ch. 8; pp.217--270) from ]{BoxSteffensmeieretal2008} % http://dx.doi.org/10.1093/oxfordhb/9780199286546.001.0001
\reading{Cartwright2007}
\reading{CartwrightHardie2012}
\reading{MorganWinship2015}


% 12
\lecture{Literature Review (Jan.~19)}{What is a ``scientific literature''? How do we know what we think we know about the social and political world? How do we evaluate, synthesize, and integrate a collective body of research evidence?}

\reading{LauRovner2009}
\reading{LeviStoker2000}
\reading{Ioannidis2005}

\seealso 

\reading{Sterling1959}
\reading{Gerberetal2010}
\reading{Schimmack2012}



% 13
\lecture{Theory Development and Hypothesis Generation (Jan.~26)}{How do we create social science theories based on past evidence and novel observation? What roles do induction and deduction play in contemporary political science? Hypotheses are the observable implications of theories. How do we derive hypotheses from theories? How do we overcome ``observational equivalence'' wherein multiple theories yield similar expectations about the world? What does it mean to test a hypothesis?}

\textbook[Ch. 3]{Toshkov}
\reading{Fearon1991}
\reading{Tannenwald1999} % Nuclear Taboo

\seealso
\reading{Fiorina1975}
%% Albert O. Hirschman, “The Search for Paradigms as a Hindrance to Understanding.” World Politics 22 (April 1970): 329-343.
%% Bueno de Mesquita, Bruce (1985) – “Toward a scientific understanding of international conflict: A personal view,” International Studies Quarterly, 29(2), 121-136.




% 14
\lecture{Case Studies and Case Comparisons (Feb.~2)}{How do comparisons between cases help us to make inferences about causality? How do we select cases so that comparisons between them are informative about theories and hypotheses?}

\textbook[Ch. 9--11]{Toshkov}
\reading{LangeMahoneyvomHau2006}
\reading{DonerRitchieSlater2005}

\seealso
\reading{Gerring2004} % What Is A Case Study and What Is It Good For?
\reading{Mershon1996}
\reading{Wedeen1998}
\reading{Gerring2012b}
\reading{MahoneyGoertz2004}
\reading{HardingFoxMehta2002}
\reading{Hacker1998}
\reading{Mahoney2000b}
\reading{Lijphart1971}
\reading{Mahoney1999} 
\reading{Geddes1991}
\reading{CollierMahoney1996}
\reading{DrezeSen1989} %  https://shibboleth2sp.sams.oup.com/Shibboleth.sso/Login?entityID=https://lse.ac.uk/idp&target=https://shibboleth2sp.sams.oup.com/shib?dest=http://www.oxfordscholarship.com/SHIBBOLETH?dest=http://dx.doi.org/10.1093/0198283652.001.0001
%\reading{Posner2004} %% Posner, Daniel. 2004. “The Political Salience of Cultural Difference: Why Chewas and Tumbukas are Allies in Zambia and Adversaries in Malawi.” American Political Science Review 98:4 (November) 529-46.


% 15
\lecture{Causal Mechanisms (Feb.~9)}{Aside from knowing that one thing (X) caused another thing (Y), we often want to know how that causal process worked. This is the study of ``causal mechanisms''. How do we study causal mechanisms to gain a deeper understanding of causal relationships in politics? How do we study the process by which a causal effect plays out?}


\reading[Ch. 10 (pp.325--353) from ]{MorganWinship2015} % EBOOK AVAILABLE
\phantom{abcde} Available online at \href{http://ebooks.cambridge.org.gate2.library.lse.ac.uk/ebook.jsf?bid=CBO9781107587991}{Cambridge Books Online}

\reading{Brady2004} %(\url{https://www.dawsonera.com/guard/protected/dawson.jsp?name=https://lse.ac.uk/idp&dest=http://www.dawsonera.com/depp/reader/protected/external/AbstractView/S9781442203457})

\seealso
\reading{RueschemeyerStephens1997}
\reading{Imaietal2011}
\reading{BullockGreenHa2010}

\subsection*{Reading Week -- No Lecture or Class (Feb.~12--16)}
\vspace{1em}


% 16
\lecture{Statistical Inference (Feb.~23)}{Random sampling allows us to quantitatively test hypotheses about empirical regularities. This allows us to make claims about ``statistical significance'' (such as whether two groups differ from one another or whether a feature of a group differs from an expectation dictated by theory). How do we use statistical significance testing in political science? How do we interpret statistical significance tests?}

No assigned readings. Lecture and class will focus on a laboratory-type activity.

\seealso
\reading{Robinson1950} %% unit of measurement
\textbook[Ch. 7]{Kellstedt and Whitten} % sampling distributions, t-test, prop-test


% 17
\lecture{Getting to Regression: The Workhorse of Quantitative Political Analysis (Mar.~2)}{By far the most commonly used method of quantitative analysis in political science is ``regression.'' What is regression? How do we use it? How do we interpret the results of regression analyses?} 

\textbook[Ch. 8]{Toshkov}
\reading{Hibbs1978}
%-- ``Babel or babble? The evolution of language.'' \textit{The Economist}, April 14, 2011. %\url{http://www.economist.com/node/18557572}

%\seealso



% 18
\lecture{Matching and Regression: Accounting for Rival Explanations (Mar.~9)}{How do we use regression analysis to make causal inferences? How do we account for the fact that an outcome we are interested in might be caused by multiple events, features, or attributes of cases?}

%\textbook[Ch.9--10]{Kellstedt and Whitten}
\reading{CusackIversenSoskice2007}

\seealso

\reading{Sekhon2009}
\reading{RosenbaumRubin1983}
\reading{AronowSamii2015}
\reading{Steineretal2010}
\reading{DehejiaWahba2002}

% 19
\lecture{Experimental Design and the Search for Quasi-Experiments (Mar.~16)}{The clearest path to causal inference is through experimentation. How does experimentation differ from observational research? Why does experimentation provide a uniquely powerful design for making causal inferences? In lieu of experimentation, how can design research around real-world variation that has quasi-experimental properties?}

\textbook[Ch. 7]{Toshkov}
\reading{GerberGreen2008}
\reading{Bhavnani2009}

\seealso 

\noindent Regarding quasi-experiments:

\reading{Dunning2012}
\reading{SekhonTitiunik2012}
\reading{BerinskyChatfield2015}
\reading{Angrist1990}

\noindent Regarding randomized experiments:

\reading{GerberGreen2012}
\reading{GlennersterTakavarasha2014}
\reading{Druckmanetal2011}



% 20
\lecture{Conclusion, Exam Prep, Synthesis (Mar.~23)}{Where have we been? What have we learned? Where do we go from here?}

\textbook[Ch. 11]{Toshkov}

\seealso
\reading{Dahl1961}
\reading{Merriam1921}
\reading{MahoneyGoertz2006}
\reading{Lowi1992}


% load bibtext, but don't generate a bibliography
\bibliographystyle{plain}
\nobibliography{Syllabus}


\end{document}
