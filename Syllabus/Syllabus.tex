\documentclass[12pt,a4paper]{article}
\usepackage[margin=1in]{geometry}
\usepackage{natbib}

\usepackage{setspace,mdwlist,comment}
\setlength{\marginparwidth}{.5in}
\setlength{\parindent}{0in}
\setlength{\parskip}{1em}

% mini table of contents
\usepackage{minitoc}
\dosecttoc % make section toc
\setcounter{secttocdepth}{2} % subsection depth
\renewcommand{\stctitle}{} % no title
\nostcpagenumbers

\usepackage{natbib}
\usepackage{bibentry}
\newcommand{\reading}[2][]{\noindent -- {#1}\bibentry{#2}.\vspace{.25em}\\}
\newcommand{\textbook}[2][]{\noindent -- {#1}#2.\vspace{.25em}\\}
\newcommand{\thomas}{\vspace{1em}\noindent Instructor: Thomas\\}
\newcommand{\seealso}{\noindent \emph{See Also:}}
\newcommand{\topic}[1]{\noindent \textbf{#1}\\}

\usepackage{hyperref}
\hypersetup{
    bookmarks=true,         % show bookmarks bar?
    unicode=false,          % non-Latin characters in Acrobat’s bookmarks
    pdftoolbar=true,        % show Acrobat’s toolbar?
    pdfmenubar=true,        % show Acrobat’s menu?
    pdffitwindow=false,     % window fit to page when opened
    pdfstartview={FitH},    % fits the width of the page to the window
    pdftitle={Syllabus: Research Design in Political Science},    % title
    pdfauthor={Thomas J. Leeper},     % author
    pdfsubject={Government} {Political Science},   % subject of the document
    pdfnewwindow=true,      % links in new window
    pdfborder={0 0 0}
}

\title{\textit{Research Design in Political Science}\\Department of Government, LSE\\2015-2016}

\begin{document}
\nobibliography*

\maketitle

\faketableofcontents

\begin{minipage}[b]{0.5\linewidth}
\textit{Instructor}\\
Thomas J. Leeper\\
Department of Government\\
Connaught House 3.21\\
\href{mailto:thosjleeper@gmail.com}{thosjleeper@gmail.com}\\
\end{minipage}
\begin{minipage}[b]{0.5\linewidth}
\textit{GTA}\\
TBD\\
\hspace{1em}\\
\hspace{1em}\\
\hspace{1em}\\
\end{minipage}


\section{Introduction}
The course will introduce students to the fundamentals of research design in political science. The course will cover a range of topics, starting from the formulation of research topics and research questions, the development of theory and empirically testable hypotheses, the design of data collection activities, and basic qualitative and quantitative data analysis techniques. 

The course will address a variety of approaches to empirical political science research including experimental and quasi-experimental designs, large-n survey research, small-n case selection, and comparative/historical comparisons. As a result, topics covered in the course will be varied and span all areas of political science including political behaviour, institutions, comparative politics, international relations, and public administration.

Every week will involve a lecture followed by a class with a Graduate Teaching Assistant. 

\clearpage
\section{Learning Objectives}
The learning objectives for the course are as follows. By the end of the course, students will be able to:

\begin{enumerate}
\item Identify interesting political science research questions and formulate theories and hypotheses that answer them
\item Describe and operationalize concepts from political science theories
\item Evaluate the strengths and weaknesses of different approaches to empirical research
\item Apply political science theories to the design of original research
\end{enumerate}

\section{Learning Assessment and Feedback}

Students will be evaluated through (1) a 2-hour written exam covering the full breadth of course content and (2) a 3000-word written paper applying course material in the form of a research design proposal. The final mark will reflect an equal weighting of both forms of assessment.

The written exam covers the full breadth of material from the course and will test students' knowledge of course content, including concept definition, the appraisal of political science theories, the generation of hypotheses, and --- most importantly --- the appropriateness of different research designs for answering specific research questions. This will count for 50\% of the final mark.

The individual research design paper should outline the basic elements of a novel research project, namely a research question, theoretical contribution, testable hypotheses, and a description of the proposed data collection and analysis. Unlike the written exam, this paper should focus narrowly on a topic of the student's choice and display a greater depth of understanding of a smaller set of ideas raised in the course. This will count for 50\% of the final mark.

As formative work in preparation for both exam forms, students will complete short ``problem set'' assignments, approximately every other week (see course schedule for details), which allow them to apply material from the course to concrete political science examples (e.g., identifying design elements of a published research paper; proposing strategies for answering a given research question, etc.). While these formative assessments do not count toward the final mark, they provide an opportunity for peer and instructor feedback.

% Problem Sets
% 1. Concept definition and measurement
% 2. [something about causality]
% 3. Analyze a set of texts
% 4. Case selection
% 5. Tabulation/Visualization activity
% 6. Survey interview
% 7. Experimental design and/or analysis
% 8. Regression
% 9. Ethics
% 10.



\clearpage
\section{Reading Material}

\section{Course Website}




\clearpage
\section{Schedule}
The general schedule for the course is as follows. Details on topics covered and the readings for each week are provided on the following pages.

\secttoc

\clearpage

% MICHAELMAS TERM

% 1
\subsection{Introduction}
\emph{}

\thomas

\subsection*{Lecture}

\begin{itemize*}
\item 
\end{itemize*}

\subsubsection*{Readings}

\seealso

% 2
\subsection{Research Questions: What do we want to know?}

% 3
\subsection{Building Theories from Observations} % Induction and Deduction

% 4
\subsection{Deriving Hypotheses from Theory} % something on research synthesis and lit reviewing?

% 5
\subsection{Concepts: ``I'll know it when I see it''}

% 6
\subsection{Measurement: Concepts in Practice}

% 7
\subsection{Causality: Explanation versus Prediction}

% 8
\subsection{Process-Tracing and Working with Texts}

% 9
\subsection{Case Selection: Comparisons over Time and Geography}

% 10
\subsection{Sampling and Representativeness}


% LENT TERM

% 11
\subsection{Probability and Statistical Inference}

% 12
\subsection{Interviewing, Structured and Unstructured}

% 13
\subsection{Questionnaire Design}

% 14
\subsection{Tabulation and Visualization}

% 15
\subsection{Matching: Accounting for Rival Explanations}

% 16
\subsection{Regression}

% 17
\subsection{Experimental Design}

% 18
\subsection{Quasi-Experiments, or the Search for Easy Answers}

% 19
\subsection{Ethics and Research Integrity}

% 20
\subsection{Conclusion and Synthesis}

% Introduction
%% Robert Dahl, "The Behavioral Approach in Political Science: Epitaph for a Monument to a Successful Protest." APSR (1961) 55:763- 72.
%% Charles Merriam, "The Present State of the Study of Politics," APSR (1921) 15:173-85.


%% Mahoney, J. & Goertz, G. 2006. "A Tale of Two Cultures: Contrasting Quantitative and Qualitative Research." Political Analysis 14: 227-249.

%% Lowi, Theodore J. 1992. “The State in Political Science: How We Become What We Study.” American Political Science Review 86, 1-7.


% Concepts
%% Goertz

%% Gerring

%% Sartori, Giovanni. 1970. “Concept Misformation in Comparative Politics.” American Political Science Review 64:4 (December) 1033-46.

%% Mansbridge, Jane. 2003. “Rethinking Representation.” American Political Science Review (November) 515-28.


% Measurement and measurement validity
%% Adcock, R. & Collier, D. 2001. “Measurement Validity: A Shared Standard for Qualitative and Quantitative Research.” American Political Science Review 95: 529-546.

%% Paxton, Pamela, “Women’s Suffrage in the Measurement of Democracy: Problems of Operationalization,” Studies in Comparative International Development 35:3 (September 2000): 92-111.
%% Munck, Gerardo L., and Jay Verkuilen, “Measuring Democracy: Evaluating Alternative Indices,” Comparative Political Studies 35:1 (February 2002): 5-34.

% Theory testing (hypothesis generation and testing; philosophy of science)
% ??? positivism
% ??? publication bias?
% ??? 


% Description
%% Athaus, Scott L., and Devon M. Largio, “When Osama Became Saddam: Origins and Consequences of the Change in America’s Public Enemy #1,” 

% Participant observation
%% Fenno, Richard F., Jr. 1977. “U.S. House Members in Their Constituencies: An Exploration.” American Political Science Review 71:3 (September) 883-917.
%% Wedeen, Lisa. 2010. “Ethnographic Work in Political Science.” Annual Review of Political Science (May).

% documents
%% Finnegan, Ruth. 1996. “Using Documents.” In Roger Sapsford and Victor Jupp (eds), Data Collection and Analysis (Thousand Oaks, CA: Sage) 138-52.
%% Lustick, Ian. 1996. “History, Historiography, and Political Science: Multiple Historical Records and the Problem of Selection Bias.” American Political Science Review (September) 605-18.


% Visualization


% Case studies
%% Gerring, John. 2004. “What Is A Case Study and What Is It Good For?” American Political Science Review. 98, 2: 341-354.
%% James Mahoney and Gary Goertz. 2004. “The Possibility Principle: Choosing Negative Cases in Comparative Research.” American Political Science Review. 98, pp. 653-669.
%% David J. Harding, Cybelle Fox, and Jal D. Mehta. 2002. “Studying Rare Events through Qualitative Case Studies: Lessons From a Study of Rampage School Shootings.” Sociological Methods & Research. 31(2): 174-217.

% External validity; Inference from sample to population
%% Increasing the Number of Observations (1994), Gary King, Robert Keohane & Sidney Verba, Designing Social Inquiry: Scientific Inference in Qualitative Research, pp. 208-230

% survey sampling and representativeness
%% papers on online panels?
% replication


% Causality
% Process-Tracing
%% Brady, Henry E., “Data-Set Observations versus Causal-Process Observations: The 2000 U.S. Presidential Election,” pp. 267-71 in Henry Brady and David Collier, eds., Rethinking Social Inquiry: Diverse Tools, Shared Standards (Lanham: Rowman and Littlefield, 2004).

%% Rueschemeyer, Dietrich, and John D. Stephens, “Comparing Historical Sequences – A Powerful Tool for Causal Analysis,” Comparative Social Research 17: 55-72.


% interviews
%% Beth Leech. 2002. “Symposium on Interview Methods in Political Science.” PS: Political Science & Politics. 23, 3: 663-688.
%% Herbert J. Rubin and Irene S. Rubin. 2005. Qualitative Interviewing: The Art of Hearing Data. Thousand Oaks, CA: Sage Publications.
%% Peabody, Robert L. et al. 1990. “Interviewing Political Elites.” PS: Political Science and Politics 23, 451-55.

% focus groups
%% Conover, Pamela Johnston, Ivor M. Crewe, and Donald Searing. 1991. “The Nature of Citizenship in the United States and Great Britain: Empirical Comments on Theoretical Themes.” Journal of Politics 53:3 (August) 800-32.



% Comparative-historical methods
%% Posner, Daniel. 2004. “The Political Salience of Cultural Difference: Why Chewas and Tumbukas are Allies in Zambia and Adversaries in Malawi.” American Political Science Review 98:4 (November) 529-46.

%% Mahoney, James “Strategies of Causal Inference in Small-N Analysis,” Sociological Methods and Research 28: 4 (May 2000), pp. 387-424.

%% Lijphart, Arend, “Comparative Politics and the Comparative Method,” American Political Science Review 65: 3 (Sept. 1971): 682-93.

%% Dreze, Jean, and Amartya Sen, “China and India,” in Dreze and Sen, Hunger and Public Action (New York: Oxford University Press, 1989).

%% James Mahoney, “Nominal, Ordinal, and Narrative Appraisal in Macrocausal Analysis,” American Journal of Sociology 104:4 (January 1999), pp. 1154-1169. 

%% Doner, Richard F., Bryan K. Ritchie, and Dan Slater, “Systemic Vulnerability and the Origins of Developmental States: Northeast and Southeast Asia in Comparative Perspective,” International Organization 59 (Spring 2005), pp. 327-361.

%% Geddes, Barbara. 1990. “How the Cases You Choose Affect the Answers the Answers You Get: Selection Bias in Comparative Politics.” Political Analysis 2: 131-150.

% Large-n Observational methods (matching)
%% Morgan and Winship?

% OVB/conditioning --> Experiments
%% Gerber, A. S. & Green, D. P. 2008. “Field Experiments and Natural Experiments.” In Box-Steffensmeier, J. M.; Brady, H. E. & Collier, D. (Eds.), Oxford Handbook of Political Methodology, Oxford University Press.

% Quasi-Experiments (ITS/DID/RDD)
%% Connecticut Crackdown on Speeding

% Correlation and regression for causal inference

% Repeated cross-sections and panel designs 



% Theory development
%% Morris P. Fiorina, “Formal Models of Political Science.” American Journal of Political Science 19 (February 1975): 133-159. 

%% Albert O. Hirschman, “The Search for Paradigms as a Hindrance to Understanding.” World Politics 22 (April 1970): 329-343.

% Explanation versus prediction
%% 1. Shmueli, Galit. 2010. “To Explain or to Predict?” Statistical Science 25(3): 289-310.
%% 2. Stevens, Jacqueline. 2012. “Political Scientists are Lousy Forecasters.” New York Times, Sunday June 26th, 2012. http://goo.gl/Iiq03L.
%% 3. Dickinson, Matthew. 2012. “No, Political Scientists are NOT Lousy Forecasters – In fact, they are Pretty Good.” Presidential Power: A Nonpartisan Analysis of Presidential Politics, June 26th, 2012. URL: http://goo.gl/8O8j3N.
%%  4. Henry Farrell. June 24, 2012. “Why the Stevens Op-Ed is Wrong.” The Monkey Cage. URL: http://themonkeycage.org/blog/2012/06/24/why-the-stevens-op-ed-is-wrong/. 

% Developing theories
%% Gerring

% Formal theory (and EITM)
% Deducing hypotheses from other kinds of theories
%% Fearon, James D. 1991. “Counterfactuals and Hypothesis Testing in Political Science.” World Politics. 43:169-95.
%% Bueno de Mesquita, Bruce (1985) – “Toward a scientific understanding of international conflict: A personal view,” International Studies Quarterly, 29(2), 121-136.

% ethics





% load bibtext, but don't generate a bibliography
\bibliographystyle{plain}
\nobibliography{Syllabus}


\end{document}
