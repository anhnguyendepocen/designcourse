\documentclass[a4, 12pt]{article}
\usepackage[top=0cm, bottom=2cm, left=2cm, right=2cm]{geometry}

\title{Concepts and measurement}
\author{}
\date{}

\begin{document}

\maketitle
\vspace{-6em}

\section{Purpose}
The purpose of this assignment is to assess your ability to define and measure political science concepts.

\section{Overview}

You are being given a concept to think about. For this concept, you need to define the concept, operationalize it, and consider cases that fall inside and outside the conceptual definition.

\section{Your Task}

\begin{enumerate}\itemsep1em
\item Consider the concept of ``terrorism''. Using the classical approach to concept definition, provide a dictionary definition of this concept. Are there any (approxiate) synonyms or antonyms for this concept? If so, list them and briefly describe why they those labels are (or are not) appropriate labels for this concept.

\item Using the family resemblance approach, list the essential, constitutive dimensions of this concept. These should be necessary and/or sufficient features.

\item Operationalize this concept. How would you measure it? How would you score an observation on this variable? Is the variable binary (``terrorism'' or ``not-terrorism'') or is it categorical or continuous? Justify your operationalization.

\item Identify two observable instances (cases) of this concept and two instances that do not fit this concept but may be closely related to the concept. Briefly explain and justify why each case falls within or outside the scope of the concept.

\item In a few sentences, briefly reflect on what was easy or difficult about this assignment and provide any feedback on the assignment for the instructor.

\end{enumerate}

\section{Submission Instructions}

Your assignment should be no more than 2 pages of double-spaced A4, Times New Roman font size 12. Submit the assignment via Moodle by Tuesday, October 27.

\section{Feedback}

You will receive feedback within two weeks.

\end{document}
