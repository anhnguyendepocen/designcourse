\documentclass[17pt]{beamer} %Makes presentation
%\documentclass[handout, 17pt]{beamer} %Makes Handouts
\documentclass[17pt]{beamer} %Makes presentation
%\documentclass[handout]{beamer} %Makes Handouts
\usetheme{Singapore} %Gray with fade at top
\useoutertheme[subsection=false]{miniframes} %Supppress subsection in header
\useinnertheme{rectangles} %Itemize/Enumerate boxes
\usecolortheme{seagull} %Color theme
\usecolortheme{rose} %Inner color theme

\definecolor{light-gray}{gray}{0.75}
\definecolor{dark-gray}{gray}{0.55}
\setbeamercolor{item}{fg=light-gray}
\setbeamercolor{enumerate item}{fg=dark-gray}

\setbeamertemplate{navigation symbols}{}
%\setbeamertemplate{mini frames}[default]
%\setbeamercovered{dynamics}
\setbeamerfont*{title}{size=\Large,series=\bfseries}
\setbeamerfont{footnote}{size=\tiny}

%\setbeameroption{notes on second screen} %Dual-Screen Notes
%\setbeameroption{show only notes} %Notes Output

\setbeamertemplate{frametitle}{\vspace{.5em}\bfseries\insertframetitle}
\newcommand{\heading}[1]{\noindent \textbf{#1}\\ \vspace{1em}}

\usepackage{bbding,color,multirow,times,ccaption,tabularx,graphicx,verbatim,booktabs}
\usepackage{colortbl} %Table overlays
\usepackage[english]{babel}
%\usepackage[latin1]{inputenc}
%\usepackage[T1]{fontenc}
\usepackage{lmodern}

%\author[]{Thomas J. Leeper}
\institute[]{
  \inst{}%
  Department of Government\\London School of Economics and Political Science
}

\usepackage{tikz}
\usetikzlibrary{shapes,arrows}

\usepackage{multicol}

\title{Literature Review}


\date[]{}

\begin{document}

\frame{\titlepage}

\frame{\tableofcontents}

\section[MT]{Brief Review of MT Material}
\frame{\tableofcontents[currentsection]}

\frame{
\frametitle{Administrative Matters}

\begin{itemize}
\item<2-> NSS
\item<3-> Research Design Proposal
\item<4-> Problem sets
	\begin{itemize}
	\item PS 6: Covers next 2 weeks
	\item PS 7: Choose an article from another course
	\item PS 8: R-based activity
	\end{itemize}
\end{itemize}

}

\frame{

\frametitle{The Exam!}

The exam has three parts:

\begin{enumerate}
\item Short-answer questions
\item Essay analysing/evaluating an empirical article
\item Research proposal section
\end{enumerate}

Sample paper is on Moodle.

}


\frame{

\frametitle{Part B Readings}

\footnotesize

\begin{multicols}{2}
\begin{itemize}
\item Munck and Verkuilen (2002) (MT3)
\item Young and Soroka (2012) (MT7)
\item Goffman (2009) (MT8)
\item Mershon (1996) (MT9)
\item Wedeen (1998) (MT9)
\item Tannenwald (1999) (LT3)
\item Lange, Mahoney, vom Hau (2006) (LT4)
\item Doner, Ritchie, Slater (2005) (LT4)
\item Brady (2004) (LT5)
\item Hibbs (1978) (LT8)
\item Cusack, Iversen, Soskice (2007) (LT9)
\item Bhavnani (2009) (LT10)
\item Campbell and Ross (1968) (LT10)
\end{itemize}
\end{multicols}

}


\frame{}

\frame{\huge\vskip20pt\textbf{What did we learn about during MT?}}

% research questions; what makes questions interesting
% concept definition
% operationalization
% what does it mean to describe?
% texts as data
% interviews as data
% what are ``cases''?
% representativeness
% ethics


\section{Brief Review of Last Week}
\frame{\tableofcontents[currentsection]}

\frame{\centering\huge\vskip20pt
\textbf{What does it mean to think counterfactually?}
}

\frame{\centering\huge\vskip20pt
\textbf{How does doing so help us make causal inferences?}
}

% we skipped some material on experiments; we'll get back to it

\frame{}


\section{Scientific ``Literatures''}
\frame{\tableofcontents[currentsection]}

\frame{

\frametitle{What is a literature?}

\begin{itemize}\itemsep1em
\item Definition: A accumulated body of written work that collectively constitutes knowledge in a specific field of study.
	\begin{itemize}
	\item Basically: \textit{what we know and don't know}
	\item In philosophy, sometimes ``the canon''
	\end{itemize}
\item<2-> All research builds on ``the literature''
\item<3-> All research should contribute to ``the literature''
\end{itemize}
}

\frame{

\frametitle{What is a literature?}

\begin{itemize}\itemsep1em
\item<2-> How do we decide what individual pieces of research fall within ``the literature''?
\item<3-> Any literature is amorphous and ultimately individually and socially constructed
	\begin{itemize}
	\item<4-> ``The literature'' is what is relevant to your research
	\item<5-> Others may disagree with your definition of what research is relevant versus irrelevant
	\end{itemize}
\end{itemize}

}

\frame{

\frametitle{Organizing Literature}

There are a few broad ways that we might identify ``a literature'':

\begin{enumerate}
\item Research using shared concepts
\item Research using shared theory
\item Research using shared data sources
\item Research using shared methods of analysis
\item Research by the same author(s)/team(s)
\end{enumerate}

}

\frame{
\frametitle{I. Concepts}

\begin{itemize}\itemsep0.5em
\item Studies on a common concept, e.g.:
	\begin{itemize}
	\item Trust
	\item Negative advertising
	\item Economic growth
	\item Democratization
	\item Justice
	\item etc.
	\end{itemize}
\item<2-> Individual studies may have little in common except for the concept at focus in the study
\end{itemize}

}


\frame{
\frametitle{II. Theory}

\begin{itemize}\itemsep0.5em
\item Studies working from a given theoretical perspective, e.g.:
	\begin{itemize}
	\item Rational choice
	\item Marxism
	\item Feminism
	\item<2-> Epigenetics
	\item<2-> Prospect theory
	\item<2-> Theory of Planned Behaviour
	\end{itemize}
\item<3-> Individual studies may have little in common except for the broad theory stance
\end{itemize}

}

\frame{
\frametitle{III. Data}

\begin{itemize}\itemsep0.5em
\item Studies working with a particular type or source of data, e.g.:
	\begin{itemize}
	\item Area Studies
		\begin{itemize}
		\item British Politics
		\item African Politics
		\end{itemize}
	\item The 2015 British Election Study
	\item The Comparative Manifesto Dataset
	\end{itemize}
\item<2-> Individual studies may have little \textit{substantively} in common
\end{itemize}

}

\frame{
\frametitle{IV. Methods}

\begin{itemize}\itemsep0.5em
\item Studies working with particular methods, e.g.:
	\begin{itemize}
	\item Ethnography
	\item Text analysis
	\item Experimentation
	\item Elite interviewing
	\item Surveys
	\end{itemize}
\item<2-> Individual studies may have little in common except empirics
	\begin{itemize}
	\item Often norms or ``best practices'' in the application of particular methods, regardless of research context
	\end{itemize}
\end{itemize}

}

\frame{
\frametitle{V. Authors}

\small

\begin{itemize}\itemsep0em
\item Studies conducted by a given author or network of authors, e.g.:
	\begin{itemize}
	\item<2-> Kahneman and Tversky
	\item<3-> Mansbridge
	\item<4-> LIGO collaboration
	\item<5-> The Sidanius Lab
	\item<6-> ``The Michigan School''
	\end{itemize}
\item<7-> Often an author or team will produce multiple works on a theme over time, using common concepts, theory, methods, and data
\item<8-> Rivalries!
\end{itemize}

}


\frame{

\frametitle{Putting it all together}

\begin{itemize}\itemsep1em
\item Think of these organizing frameworks like a Venn Diagram, where each feature can overlap
\item A literature is the subset of the complete diagram that is relevant to a particular piece of research
\end{itemize}

}

\frame{}

\frame{

\frametitle{Finding Literature}

\large
\begin{center}
Question: How do you find literature?
\end{center}
}

\frame{

\frametitle{Finding Literature}
\begin{itemize}\itemsep0.5em
\item Library or Google Scholar search
\item Talk to faculty members and peers
\item Research syntheses
\item Journals
	\begin{itemize}
	\item Generalist
	\item Subfield
	\end{itemize}
\item Citation networks
\end{itemize}
}

\frame{

\frametitle{Citation Networks}

\small

\begin{itemize}\itemsep0.25em
\item A citation network is the set of unidirectional connections formed by ``co-citation'' (i.e. one piece of research citing another piece of research)

\item<2-> Citations reflect:
	\begin{itemize}
	\item<3-> Authors' positioning a piece of research within a literature
	\item<4-> ``Positive'' citation to research they wish to expand upon, elaborate, or praise
	\item<5-> ``Negative'' citation to research they wish to criticize
	\end{itemize}
\end{itemize}

}

\frame{

\frametitle{Citation Networks}

\small

Problems with using citation networks to understand a literature include:

\begin{itemize}\itemsep0.25em
\item<2-> Limited numbers of citations (can't cite everything!)
\item<3-> Positive/negative citation ambiguity
	\begin{itemize}
	\item ``Network centrality'' only reflect volume of use, not quality
	\end{itemize}
\item<4-> Intentional omission of relevant research
\item<5-> Self-citation
\item<6-> \textit{Forward} citation search can be difficult
\end{itemize}

}


\frame{}


\frame{

\frametitle{Research Synthesis}

\begin{itemize}\itemsep0.5em
\item Definition: A research synthesis is a formal review of existing literature that accumulates evidence from multiple studies.
	\begin{itemize}
	\item<2-> \textit{Qualitative} literature review
	\item<2-> \textit{Quantitative} literature review (meta-analysis)
	\end{itemize}
\item<3-> Serves two functions:
	\begin{itemize}
	\item Summary of existing knowledge
	\item Identification of limitations or gaps
	\end{itemize}
\end{itemize}

}

\frame{

\frametitle{Qualitative Review}

\begin{itemize}\itemsep0.25em
\item A narrative summary of a body of research, organized around a concept, a theory, a method, a data source, or an author
	\begin{itemize}
	\item Usually covers a specific period of time
	\item May focus on studies from a particular context, or particular perspective
	\end{itemize}
\item<2-> Structure and content is flexible
	\begin{itemize}
	\item May merely summarize
	\item May be ``critical''
	\end{itemize}
\end{itemize}

}

% This is what you should do for your research proposal.



\frame{

\frametitle{Quantitative Review}

\begin{itemize}\itemsep1em
\item A numerical summary of a body of research, organized around a specific statistic of interest (e.g., a prevalence, a correlation, a causal effect)
	\begin{itemize}
	\item Also called ``meta-analysis''
	\end{itemize}
\item<2-> Structure and content is not flexible
\item<3-> Provides a specific inference with the intent to guide new research or inform policy
\end{itemize}

}

\frame{

\frametitle{Quantitative Review}

Structure is nearly always the same:

\small
\begin{enumerate}\itemsep0em
\item A body of existing studies is gathered using inclusion/exclusion criteria
\item Statistic of interest is derived for each study
\item Statistics are mathematically aggregated (by some form of weighted averaging)
\item Patterns are examined across between-study sources of variation
\item An overall estimate is produced
\end{enumerate}

}



\frame{

\frametitle{Qualitative versus Quantitative Synthesis}

\small

\begin{itemize}\itemsep0.25em 
\item Remember: All quantitative synthesis is also qualitative
\item Solely qualitative synthesis will tend to be more holistic
\item Solely quantitative synthesis will tend to provide less critical engagement with specific aspects of individual studies
\item Quantitative synthesis typically makes evidence selection criteria more explicit
\end{itemize}

}

\frame{}

\frame{

\frametitle{Evaluating Research}

\begin{center}
In groups of 3, try to generate a set of criteria that you can use to decide whether to believe a given piece of research.

\vspace{1em}

You have 6 minutes.

\end{center}

}

\frame{}


\frame<1-4>[label=yourproject]{

\frametitle{{\normalsize Your Research}}

\small

\begin{enumerate}\itemsep0em
\item<2-> All research should start from a literature
\item<3-> Identify which literature or literatures you want to address
\item<4-> Develop a critical literature review of existing evidence
	\begin{itemize}
	\item Identify ways you could contribute new knowledge
	\end{itemize}
\item<5-> Develop theory and design empirical methods using and/or improving existing theory and methods
	\begin{itemize}
	\item Focus on improving \textbf{one} thing
	\end{itemize}
\end{enumerate}

}

\frame{

\frametitle{{\normalsize How does research add to a literature?}}

\small

\begin{enumerate}
\item<2-> Answering unanswered questions
\item<3-> Applying theory to new cases
\item<4-> Elaborating more specific implications of theories
\item<5-> Developing new measures of concepts
\item<6-> Developing new or better concept definitions
\item<7-> New, improved, and/or alternative methodology
\item<8-> Studying alternative units of analysis
\end{enumerate}

}


\againframe<4->{yourproject}


\frame{

\centering
\huge
Questions?

}

\frame{}


\end{document}
