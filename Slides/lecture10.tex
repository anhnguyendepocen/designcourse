\documentclass[17pt]{beamer} %Makes presentation
%\documentclass[handout]{beamer} %Makes Handouts
\usetheme{Singapore} %Gray with fade at top
\useoutertheme[subsection=false]{miniframes} %Supppress subsection in header
\useinnertheme{rectangles} %Itemize/Enumerate boxes
\usecolortheme{seagull} %Color theme
\usecolortheme{rose} %Inner color theme

\definecolor{light-gray}{gray}{0.75}
\definecolor{dark-gray}{gray}{0.55}
\setbeamercolor{item}{fg=light-gray}
\setbeamercolor{enumerate item}{fg=dark-gray}

\setbeamertemplate{navigation symbols}{}
%\setbeamertemplate{mini frames}[default]
%\setbeamercovered{dynamics}
\setbeamerfont*{title}{size=\Large,series=\bfseries}
\setbeamerfont{footnote}{size=\tiny}

%\setbeameroption{notes on second screen} %Dual-Screen Notes
%\setbeameroption{show only notes} %Notes Output

\setbeamertemplate{frametitle}{\vspace{.5em}\bfseries\insertframetitle}
\newcommand{\heading}[1]{\noindent \textbf{#1}\\ \vspace{1em}}

\usepackage{bbding,color,multirow,times,ccaption,tabularx,graphicx,verbatim,booktabs}
\usepackage{colortbl} %Table overlays
\usepackage[english]{babel}
%\usepackage[latin1]{inputenc}
%\usepackage[T1]{fontenc}
\usepackage{lmodern}

%\author[]{Thomas J. Leeper}
\institute[]{
  \inst{}%
  Department of Government\\London School of Economics and Political Science
}

\usepackage{tikz}
\usetikzlibrary{shapes,arrows}
\usepackage[normalem]{ulem}

\title{Translating Texts into Interpretations and Numbers}

% Primary and secondary source documents provide a written record of politically relevant events and processes. Texts can be used in a number of ways in political science research. How do we draw meaning from texts in qualitative and quantitative ways? How does textual information become useful data for making political inferences?



\date[]{}

\begin{document}

\frame{\titlepage}

\frame{\tableofcontents}

\frame{

\frametitle{Preview}

\begin{itemize}
\item Three weeks on data collection
	\begin{itemize}
	\item Text analysis
	\item Interviewing
	\item Participant Observation
	\end{itemize}
\item Problem Set 4
	\begin{itemize}
	\item Due December 15
	\item Team assignment
	\end{itemize}
\end{itemize}

}


\section{Texts as Sources}
\frame{\tableofcontents[currentsection]}


\frame{

\frametitle{What counts as text?}

\begin{itemize}\itemsep1em
\item Tertiary sources
\item Secondary sources
\item Primary sources
\end{itemize}

}

% brainstorm examples


\frame{

\frametitle{How do you use texts?}

\begin{itemize}\itmesep1em
\item Think about your own experience reading, interpreting, and interacting with textual sources for academic purposes (e.g, for writing a term paper).
\item With the person sitting next to you, discuss:
	\begin{enumerate}
	\item The process by which you try to understand the meaning and content of texts
	\item How you choose texts to read
	\end{enumerate}
\end{itemize}

}



% texts in lieu of observation
% we cannot see everything ourselves, so we rely on evidence

% texts as CPOs vs. DSOs


% selection bias -> when do we stop looking for evidence?
% -> how do we know that we have all of the evidence?
% -> If we go looking for information about a case as something, do we miss evidence that sees that case as a case of something else?

% define historiography

% texts are subjective


% research questions and text: 
%% at one level, texts can be used to answer any research question
%% there are, however, a lot of interesting research questions that are about text per se

% what is the sentiment of this text?
% what is the ideology of this text?
% what discourses or frames does this text communicate?
% what issues does this text mention?
% how does this text differ from other texts (by other actors, from other time periods)?
% what do these several texts have in common?
% can we categorize texts to reveal something about their authors (e.g., their ideology, their emotional state, their intelligence)?


% unit of analysis???
% text, paragraph, sentence
% multiple texts that represent some larger unit (speaker/source, country, party)


\section{Content Analysis}
\frame{\tableofcontents[currentsection]}

% definition: systematic description of the content of a communication

% can be applied to any kind of document (primary, secondary, tertiary)
% need not be textual (we can content analyze images, videos, audio, actions/behaviors/expressions)
% graph showing textual input of one or more strings of text into an output of numbers/scores


% what are the variables? what do we want to know about this text?


% scoring: ideology, positivity/negativity (both rely on word associations)
% dictionary methods (sentiment: lexicoder, ideology: wordfish)

% sentiment
If only all politicians could \textit<3->{believe} in Britain as UKIP does. If only they could \textit<3->{share} our \textbf<2->{positive} vision of Britain as a \textbf<2->{proud}, \textit<3->{independent} \textit<3->{sovereign} nation, a country \textbf<2->{respected} on the world stage, a \textit<3->{major} player in global trade, with \textbf<2->{influence} and \textit<3->{authority} when it comes to tackling the pressing international issues of the day.

% ideology





% categorization: 
% identification of frames, arguments, discourse

If only all politicians could \textbf<2>{believe in Britain} as UKIP does. If only they could share our \textbf<2>{positive vision of Britain} as a \textbf<2>{proud}, \textbf<3-4>{independent sovereign nation}, a \textbf<4>{country respected on the world stage}, a \textbf<4>{major player in global trade}, with \textbf<4>{influence and authority} when it comes to tackling the pressing international issues of the day.

% discourses/frames
% 1: national pride/patriotism
% 2: independence
% 3: influential world power
% maybe others

% tabulating persons, issues, mentions
% necessarily iterative --> develop a set of categories as they are identified; requires revisiting texts


% persons
If only all \textbf<2>{politicians} could believe in Britain as \textbf<2>{UKIP} does. If only they could share our positive vision of Britain as a proud, independent sovereign nation, a country respected on the world stage, a major player in global trade, with influence and authority when it comes to tackling the pressing international issues of the day.


% issue mentions
\sout<2->{If only all politicians could believe in Britain as UKIP does.}
\sout<3->{If only they could share our positive vision of Britain}
\sout<4->{as a proud,} \textbf<4->{independent} \sout<4->{sovereign nation,}
\sout<5->{a country respected on the world stage,}
\sout<6->{a major player in} \textbf<6->{global trade}, 
\sout<7->{with influence and authority}
\sout<9->{when it comes to tackling the \textit<8>{pressing international issues} of the day.}





% sentence
If only all politicians could believe in Britain as UKIP does.
If only they could share our positive vision of Britain as a proud, independent sovereign nation, a country respected on the world stage, a major player in global trade, with influence and authority when it comes to tackling the pressing international issues of the day.

% sentence fragments
If only all politicians could believe in Britain as UKIP does.
If only they could share our positive vision of Britain 
as a proud, independent sovereign nation, 
a country respected on the world stage, 
a major player in global trade, 
with influence and authority 
when it comes to tackling the pressing international issues of the day.



% "bag of words" methods

\sout<2->{If} only all politicians could believe \sout<2->{in} Britain \sout<2->{as} UKIP does \sout<2->{.} \sout<2->{If} only they could share our positive vision \sout<2->{of} Britain \sout<2->{as} \sout<2->{a} proud\sout<2->{,} independent sovereign nation\sout<2->{,} \sout<2->{a} country respected \sout<2->{on} \sout<2->{the} world stage\sout<2->{,} \sout<2->{a} major player \sout<2->{in} global trade\sout<2->{,} \sout<2->{with} influence \sout<2->{and} authority \sout<2->{when} \sout<2->{it} comes \sout<2->{to} tackling \sout<2->{the} pressing international issues \sout<2->{of} \sout<2->{the} day\sout<2->{.}

only all politicians could believe Britain UKIP does only they could share our positive vision Britain proud independent sovereign nation country respected world stage major player global trade influence authority comes tackling pressing international issues day

independent sovereign nation country international world global 
positive proud respected major influence authority 
politicians player Britain Britain UKIP 
believe share tackling pressing 
could could does comes 
vision stage day trade issues 
only only all they our 

% term-document matrix


% unigrams, bigrams, trigrams, n-grams
% roots: family, families, families’, and familial all become famili.

% multiword expressions
% example: ``If only'' has a negative connotation even though ``if'' and ``only'' are basically neutral words on their own


% creating a codebook
% what variables do you want to measure?
% how do you measure them? at what unit of analysis?


% training set and test set; human coding and automated coding




\section{Group Discussions}
\frame{\tableofcontents[currentsection]}


\frame{

\frametitle{Problem Set 4}

\small

\begin{enumerate}\itemsep0.5em
\item You have been assigned teams
\item Develop a research question that can be answered using the party manifestos from the UK 2015 General Election
\item Develop a content analyze scheme or codebook and use it to analyze the manifestos
\item Write up your results as a group
\end{enumerate}

}


\frame{

\frametitle{Group Time}

With remaining time, gather in your Problem Set 4 groups and discuss the assignment.\\

Let me know via email or office hours if you have questions between now and the due date.

}


\frame{

\vspace{3em}

\begin{center}
Enjoy your holiday break!
\end{center}


}



\appendix
\frame{}

\end{document}
