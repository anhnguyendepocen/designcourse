\documentclass[17pt]{beamer} %Makes presentation
%\documentclass[handout]{beamer} %Makes Handouts
\usetheme{Singapore} %Gray with fade at top
\useoutertheme[subsection=false]{miniframes} %Supppress subsection in header
\useinnertheme{rectangles} %Itemize/Enumerate boxes
\usecolortheme{seagull} %Color theme
\usecolortheme{rose} %Inner color theme

\definecolor{light-gray}{gray}{0.75}
\definecolor{dark-gray}{gray}{0.55}
\setbeamercolor{item}{fg=light-gray}
\setbeamercolor{enumerate item}{fg=dark-gray}

\setbeamertemplate{navigation symbols}{}
%\setbeamertemplate{mini frames}[default]
%\setbeamercovered{dynamics}
\setbeamerfont*{title}{size=\Large,series=\bfseries}
\setbeamerfont{footnote}{size=\tiny}

%\setbeameroption{notes on second screen} %Dual-Screen Notes
%\setbeameroption{show only notes} %Notes Output

\setbeamertemplate{frametitle}{\vspace{.5em}\bfseries\insertframetitle}
\newcommand{\heading}[1]{\noindent \textbf{#1}\\ \vspace{1em}}

\usepackage{bbding,color,multirow,times,ccaption,tabularx,graphicx,verbatim,booktabs}
\usepackage{colortbl} %Table overlays
\usepackage[english]{babel}
%\usepackage[latin1]{inputenc}
%\usepackage[T1]{fontenc}
\usepackage{lmodern}

%\author[]{Thomas J. Leeper}
\institute[]{
  \inst{}%
  Department of Government\\London School of Economics and Political Science
}

\usepackage{tikz}
\usetikzlibrary{shapes,arrows}

\title{Translating Texts into Interpretations and Numbers}

% Primary and secondary source documents provide a written record of politically relevant events and processes. Texts can be used in a number of ways in political science research. How do we draw meaning from texts in qualitative and quantitative ways? How does textual information become useful data for making political inferences?



\date[]{}

\begin{document}

\frame{\titlepage}

\frame{\tableofcontents}

\frame{

\frametitle{Preview}

\begin{itemize}
\item Three weeks on data collection
	\begin{itemize}
	\item Text analysis
	\item Interviewing
	\item Participant Observation
	\end{itemize}
\item Problem Set 4
	\begin{itemize}
	\item Due December 15
	\item Team assignment
	\end{itemize}
\end{itemize}

}


\section{Texts as Sources}
\frame{\tableofcontents[currentsection]}


\frame{

\frametitle{What counts as text?}

\begin{itemize}\itemsep1em
\item Tertiary sources
\item Secondary sources
\item Primary sources
\end{itemize}

}

% brainstorm examples


\frame{

\frametitle{How do you use texts?}

\begin{itemize}\itmesep1em
\item Think about your own experience reading, interpreting, and interacting with textual sources for academic purposes (e.g, for writing a term paper).
\item With the person sitting next to you, discuss:
	\begin{enumerate}
	\item The process by which you try to understand the meaning and content of texts
	\item How you choose texts to read
	\end{enumerate}
\end{itemize}

}



% texts in lieu of observation
% we cannot see everything ourselves, so we rely on evidence



% selection bias -> when do we stop looking for evidence?
% -> how do we know that we have all of the evidence?
% -> If we go looking for information about a case as something, do we miss evidence that sees that case as a case of something else?

% define historiography

% texts are subjective




\section{Content Analysis}
\frame{\tableofcontents[currentsection]}


% unit of analysis???
% text, paragraph, sentence
% multiple texts that represent some larger unit (speaker/source, country, party)


% scoring: ideology, positivity/negativity (both rely on word associations)
% dictionary methods (sentiment: lexicoder, ideology: wordfish)
% training set and test set; human coding and automated coding


% creating a codebook




% categorization: 
% identification of frames, arguments, discourse
% tabulating persons, issues, mentions
% necessarily iterative --> develop a set of categories as they are identified; requires revisiting texts



% issue mentions
% sentence fragments


% "bag of words" methods
% unigrams, bigrams, trigrams, n-grams
% roots: family, families, families’, and familial all become famili.

% term-document matrix



\section{Group Discussions}
\frame{\tableofcontents[currentsection]}

\frame{

\frametitle{Group Time}

With remaining time, gather in your Problem Set 4 groups and discuss the assignment.\\

Let me know via email or office hours if you have questions between now and the due date.

}


\frame{

\vspace{3em}

\begin{center}
Enjoy your holiday break!
\end{center}


}



\appendix
\frame{}

\end{document}
