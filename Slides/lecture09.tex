\documentclass[17pt]{beamer} %Makes presentation
%\documentclass[handout]{beamer} %Makes Handouts
\usetheme{Singapore} %Gray with fade at top
\useoutertheme[subsection=false]{miniframes} %Supppress subsection in header
\useinnertheme{rectangles} %Itemize/Enumerate boxes
\usecolortheme{seagull} %Color theme
\usecolortheme{rose} %Inner color theme

\definecolor{light-gray}{gray}{0.75}
\definecolor{dark-gray}{gray}{0.55}
\setbeamercolor{item}{fg=light-gray}
\setbeamercolor{enumerate item}{fg=dark-gray}

\setbeamertemplate{navigation symbols}{}
%\setbeamertemplate{mini frames}[default]
%\setbeamercovered{dynamics}
\setbeamerfont*{title}{size=\Large,series=\bfseries}
\setbeamerfont{footnote}{size=\tiny}

%\setbeameroption{notes on second screen} %Dual-Screen Notes
%\setbeameroption{show only notes} %Notes Output

\setbeamertemplate{frametitle}{\vspace{.5em}\bfseries\insertframetitle}
\newcommand{\heading}[1]{\noindent \textbf{#1}\\ \vspace{1em}}

\usepackage{bbding,color,multirow,times,ccaption,tabularx,graphicx,verbatim,booktabs}
\usepackage{colortbl} %Table overlays
\usepackage[english]{babel}
%\usepackage[latin1]{inputenc}
%\usepackage[T1]{fontenc}
\usepackage{lmodern}

%\author[]{Thomas J. Leeper}
\institute[]{
  \inst{}%
  Department of Government\\London School of Economics and Political Science
}

\usepackage{tikz}
\usetikzlibrary{shapes,arrows}

\title{Causal Mechanisms and Process-Tracing}

% Aside from knowing that one thing (X) caused another thing (Y), we often want to know how that causal process worked. This is the study of ``causal mechanisms''. How do we study causal mechanisms to gain a deeper understanding of causal relationships in politics? How do we study the process by which a causal effect plays out?


\date[]{}

\begin{document}

\frame{\titlepage}

\frame{\tableofcontents}


\section{Review}
\frame{\tableofcontents[currentsection]}

\frame{
	\frametitle{Review Case Studies}
	\begin{itemize}
	\item Many uses of case studies
	\item In case comparisons (last week), we focused on scoring cases on variables to test theories \textit{between cases}
	\item Now we focus on \textit{within-case} comparisons
	\end{itemize}
}


\frame{
\frametitle{{\large Causal Process Observations}}

\normalsize

\begin{itemize}\itemsep0.5em
\item Definition: ``An insight or piece of data that provides information about the context, process, or mechanism, and that contributes distinctive leverage in causal inference''\footnote{Brady and Collier 2004, p.277}
\item Might be used to:
	\begin{itemize}
	\item Inductively generate hypotheses about potential causal relationships
	\item Deductively test a chain of causal relationships
	\end{itemize}
\end{itemize}
}


\frame{}

\section{Mechanisms}
\frame{\tableofcontents[currentsection]}


\frame{
\frametitle{Four (or five) principles of causality\footnote{From Kellstedt and Whitten}}
\begin{enumerate}
\item Correlation
\item Nonconfounding
\item Direction (``temporal precedence'')
\item \textbf<2->{Mechanism}
\item (Appropriate level of analysis)
\end{enumerate}
}


\frame{
\frametitle{Mediators/Mechanisms}

\begin{itemize}
\item Definition 1: ``the generative mechanism through which the focal independent variable is able to influence the dependent variable of interest'' (Baron and Kenny 1986, 1173)\footnote{Baron, R.M., and Kenny, D.A. 1986. ``The Moderator-Mediator Variable Distinction in Social Psychological Research: Conceptual, Strategic, and Statistical Considerations.'' \textit{Journal of Personality and Social Psychology} 51(6): 1173--1182.\par}
\item Dropping the tautology, ``the pathway(s) or process(es) by which an effect is produced''
\item Allows us to distinguish:
	\begin{itemize}
	\item \textit{Direct} effects
	\item \textit{Indirect} effects
	\end{itemize}
\end{itemize}
}
% define mechanism
% synonyms: mediator, causal pathway, ``how'' of a causal effect
% thinking about mechanisms allow us to talk about direct and ``mediated'' effects
% that is to say: how are cause and outcome linked causally?


% do we care about mechanisms? how deep do we want to go into a mechanism?
% when are we satisfied that we have ``bottomed out'' a mechanism?

% causal graphs
% causality is complicated

% if concepts have many parts and we theorize that it is one part of that concept that transmits a causal effect, should we ever study causality at the level of concepts or should we only study causality at the level of the constitutive attribute that is thought to be causally relevant?

% Morgan and winship are especially interested in situations where we don't know if X causes Y (e.g., smoking and cancer)
% but where we can observe pieces of that causal chain that might allow us to add up to a link between smoking and cancer
% Pearl's ``front door criterion'' we can learn how X affects Y if we have an exhaustive and isolated set of mechanisms

% talk about deterministic causality: each step being necessary for the next step to be possible
% versus probabilistic causality (or deterministic but heterogeneous causal effects): each step increases the value of the next step


\frame{}

\section{Process Tracing}
\frame{\tableofcontents[currentsection]}

% at its most basic level, it answers ``what happened?'' (i.e., it is descriptive history)
% the difference, however, is that it is about causal inference, which is implicitly about within-case counterfactuals
% generally, these are treated with a deterministic perspective on causality (if this hadn't happened, what would have happened instead?)
% class examples: Sherlock Holmes stories are process-tracing tests of how murders (or other crimes) that link a potential cause (a murderer) to an outcome (a death) via a series of causal steps that leave behind pieces of evidence

% application of logic and ``counterfactual'' method
% update beliefs about counterfactuals in sequence

% inductive versus deductive approach
% process tracing as a standalone method versus as a supplement to other methods
	% for example, often use very aggregated data to establish a relationship and process tracing to document how it comes about

\frame{
\frametitle{Inductive Process Tracing}

\begin{itemize}\itemsep0.5em
\item Broad search for sequential steps necessary for an event to occur
\item No \textit{a priori} expectations to test
\item Analogous to detective work
\end{itemize}
}

% example: Sherlock Holmes


\frame{
\frametitle{Deductive Process Tracing}

\begin{itemize}\itemsep0.5em
\item Sequence of within-case hypothesis tests
\item Theory or extant evidence guide chosen comparisons
	\begin{itemize}
	\item May iterate if there is no/weak evidence for one's hypothesis/es
	\end{itemize}
\end{itemize}
}

% Example: smoking and cancer


\frame{
\frametitle{{\large Four Types of Process Tracing Tests\footnote{Note: I am not a fan of this typology.}}}

Broadly consistent with Neyman-Pearson hypothesis testing.

\begin{enumerate}
\item Straw-in-the-wind test
\item Hoop test
\item Smoking gun test
\item Doubly decisive test
\end{enumerate}

}
% reason I do not like it is that it is difficult to specify a priori whether something is one of these types of tests
% part of process-tracing is not knowing what you're looking for, let alone whethere that evidence will or will not be consistent with expectations




\frame{
\frametitle{Major Caveat: Uncertainty}

\begin{itemize}\itemsep0.5em
\item Our certainty about a causal relationship is a direct function of sample size
\item Case studies methods have small sample sizes
\item Process-tracing is generally a single-case design
	\begin{itemize}
	\item Reduce uncertainty by finding within-case variation
	\item Accept only high certainty about specific case % but high uncertainty about the general class of cases of which this case is a member
	\end{itemize}
\end{itemize}

}


\section{Preview}
\frame{\tableofcontents[currentsection]}


\frame{
\frametitle{Research Design Proposal}

\begin{itemize}\itemsep0.5em
\item Instructions posted on Moodle
\item Use class sessions to discuss topics
\item Don't worry about design now
\item Focus instead on topics, questions, and theories
\end{itemize}
}


\frame{
\frametitle{Coming weeks (MT and LT)}

\begin{itemize}\itemsep0.5em
\item Methods of data collection
	\begin{enumerate}
	\item Text
	\item Interviews/Surveys
	\item Observation
	\end{enumerate}
\item Problem Set 4 (due in December)
\item Shift to methods of quantitative data analysis
\end{itemize}
}


\appendix
\frame{}

\end{document}
