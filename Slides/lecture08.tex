\documentclass[17pt]{beamer} %Makes presentation
%\documentclass[handout]{beamer} %Makes Handouts
\usetheme{Singapore} %Gray with fade at top
\useoutertheme[subsection=false]{miniframes} %Supppress subsection in header
\useinnertheme{rectangles} %Itemize/Enumerate boxes
\usecolortheme{seagull} %Color theme
\usecolortheme{rose} %Inner color theme

\definecolor{light-gray}{gray}{0.75}
\definecolor{dark-gray}{gray}{0.55}
\setbeamercolor{item}{fg=light-gray}
\setbeamercolor{enumerate item}{fg=dark-gray}

\setbeamertemplate{navigation symbols}{}
%\setbeamertemplate{mini frames}[default]
%\setbeamercovered{dynamics}
\setbeamerfont*{title}{size=\Large,series=\bfseries}
\setbeamerfont{footnote}{size=\tiny}

%\setbeameroption{notes on second screen} %Dual-Screen Notes
%\setbeameroption{show only notes} %Notes Output

\setbeamertemplate{frametitle}{\vspace{.5em}\bfseries\insertframetitle}
\newcommand{\heading}[1]{\noindent \textbf{#1}\\ \vspace{1em}}

\usepackage{bbding,color,multirow,times,ccaption,tabularx,graphicx,verbatim,booktabs}
\usepackage{colortbl} %Table overlays
\usepackage[english]{babel}
%\usepackage[latin1]{inputenc}
%\usepackage[T1]{fontenc}
\usepackage{lmodern}

%\author[]{Thomas J. Leeper}
\institute[]{
  \inst{}%
  Department of Government\\London School of Economics and Political Science
}

\usepackage{tikz}
\usetikzlibrary{shapes,arrows}

\title{Case Studies}

% Case studies are in-depth examinations of a single manifestation of a political phenomenon and are one of the most common methods of inquiry in political science. What can we do with case studies? How do they help us to understand politics?


\date[]{}

\begin{document}

\frame{\titlepage}

\frame{\tableofcontents}

\section[Elites]{Elite Interviewing}
\frame{\tableofcontents[currentsection]}

\frame{
\frametitle{Elite Interviewing}
\begin{itemize}\itemsep1em
\item Rules of questionnaire design style apply
\item Unique challenges/opportunities:
	\begin{itemize}
	\item Time constraints
	\item Guarantees of anonymity?
	\item Public information may be available
	\item Often exempt from research ethics review
	\end{itemize}
\end{itemize}
}

\frame{
	\frametitle{{\large Structured versus Unstructured}}
	\begin{itemize}\itemsep0.75em
	\item ``Structured'' interviews
		\begin{itemize}
		\item Strict order of questions
		\item Questions are precisely worded
		\item Often, closed set of response categories
		\end{itemize}
	\item<2-> Elite interviews often ``semi-structured''
		\begin{itemize}
		\item Greater use of open-ended questions
		\item More flexible ordering of questions
		\item More respondent-driven
		\end{itemize}
	\end{itemize}
}



\section{Case Studies}
\frame{\tableofcontents[currentsection]}

\frame{

\frametitle{Overview}

\begin{itemize}\itemsep0.5em
\item Consistently the most dominant method of social research
\item Often poorly executed (and poorly understood)
\item Four weeks (partially) on this topic
	\begin{itemize}
	\item Today: What is a case study?
	\item MT10: Representative sampling
	\item LT4: Case comparisons
	\item LT5: Process-tracing methods
	\end{itemize}
\end{itemize}

}

\frame{

\frametitle{What is a case study?}

\begin{itemize}\itemsep0.1em
\item Definition: ``an intensive study of a single unit for the purpose of understanding a larger class of (similar) units'' (Gerring 2004, 342)
\item Broad uses:
	\begin{itemize}
	\item Description
	\item Induction/Theory development
	\item Theory testing
	\item Exploration of mechanisms
	\item Concept definition and measurement
	\end{itemize}
\end{itemize}

}

% not simply attempting to score a case on a variable but attempting to understand it at a very deep level


% why do case studies?
%% variable identification
%% sequencing - what happened?
%% DSOs within case (different unit of analysis; or DSOs across time)
%% measurement - what is the score for this case on a variable?
%% search for confounders


% Choosing cases: purposive, representative (MT week 10), matched (next term)


% Wedeen - what is this? what is going on?
% Mershon - DSOs, lots of DSOs. what units of analysis?



\frame{

\frametitle{What counts as a case?} % activity

\begin{itemize}
\item<2-> The more important question is what is something a \textit{case of}
\item<2-> Cases are instances of a concept or phenomenon
	\begin{itemize}
	\item<3-> What is the Brexit referendum a case of?
	\item<4-> What is Islamic State a case of?
	\item<5-> What is Angela Merkel a case of?
	\item<6-> What is Sep. 11th a case of?
	\item<7-> What is Wales a case of?
	\end{itemize}
\end{itemize}

}

% common cases: countries, subnational units, persons, political parties, events (e.g., acts of terrorism), revolutions, transitions to democracy, wars, policies, legislatures, elections


% an instance or observation can be a case of many different things
% September 11th might be a case of terrorism, of structural deficiencies in buildings, of airplane hijacking, of airport security, of intelligence failure, of presidential leadership, of the onset of war, of "rallying around the flag", etc.
% to study something as a case, you need to know what it is a case *of*


\frame{
\frametitle{1: Description}
\begin{itemize}\itemsep1em
\item Case study might be descriptive
\item Historical or interpretive
\item Think ``biography''
\end{itemize}
}

% what happened; when did events happen; who was involved; etc.

\frame{
\frametitle{2: Theory development}
\begin{itemize}
\item Case is an instance of a phenomenon
\item There is some outcome to be explained
	\begin{itemize}
	\item Outcome is case itself
	\item Outcome of a case
	\item Outcome as part of case
	\end{itemize}
\item Look for ``Causal Process Observations''
\item Attempt to identify generalizable explanations
\end{itemize}
}

\frame{
\frametitle{{\large Causal Process Observations}}

\normalsize

\begin{itemize}\itemsep0.5em
\item Definition: ``An insight or piece of data that provides information about the context, process, or mechanism, and that contributes distinctive leverage in causal inference''\footnote{Brady and Collier 2004, p.277}
\item Pieces of evidence that help you inductively generate hypotheses about potential causal relationships
\end{itemize}
}

% more on this in two weeks

\frame{
\frametitle{3: Theory testing}
\begin{itemize}\itemsep1em
\item ``Actual case'' comparisons
\item Fearon's ``Counterfactual method''
\item Process tracing
\end{itemize}
}


% thinking counterfactually
	% can we observe the counterfactual (in another case, in the same case at another time, in a collection of other cases)?
	% if not, can we think about what that counterfactual might look like? 
	% see Fearon's use of ``counterfactual method''; thought experiments; hypothetical reasoning


\frame{
\frametitle{4: Mechanisms}
\begin{itemize}\itemsep0.5em
\item Imagine you already have evidence for a causal relationship
\item A case study can help you explore or test for ``mechanisms'' of that effect
\end{itemize}
}

\frame{
\frametitle{5: Concept Definition}
\begin{itemize}
\item Sometimes you don't know what you are studying
\item Case studies can clarify what something is a \textit{case of}
\item This helps you to:
	\begin{itemize}
	\item Refine your concept definition
	\item Improve measurement
	\end{itemize}
\end{itemize}
}


\frame{

\frametitle{Collection of CPOs}

\begin{itemize}
\item Qualitative analysis
	\begin{itemize}
	\item Direction observation
	\item Focus groups
	\item Interviews
	\item Archival/documentary analysis
	\end{itemize}
\item Quantitative analysis
	\begin{itemize}
	\item Surveys
	\item Experiments
	\item Statistical methods
	\item Data mining (e.g., ``big data'')
	\item Data coding
	\end{itemize}
\end{itemize}

}


\frame{}

% Questions?

\section[Activity]{Group Activity}
\frame{\tableofcontents[currentsection]}

\frame{

\frametitle{Activity}

\normalsize

\begin{itemize}
\item Think of the British Referendum on EU Membership
\item What kinds of DSOs could you collect about the referendum?
\item What other kinds of non-DSO descriptions could you make?
\item What is the referendum \textit{a case of}?
\end{itemize}

}






\appendix
\frame{}

\end{document}
