\documentclass[17pt]{beamer} %Makes presentation
%\documentclass[handout]{beamer} %Makes Handouts
\usetheme{Singapore} %Gray with fade at top
\useoutertheme[subsection=false]{miniframes} %Supppress subsection in header
\useinnertheme{rectangles} %Itemize/Enumerate boxes
\usecolortheme{seagull} %Color theme
\usecolortheme{rose} %Inner color theme

\definecolor{light-gray}{gray}{0.75}
\definecolor{dark-gray}{gray}{0.55}
\setbeamercolor{item}{fg=light-gray}
\setbeamercolor{enumerate item}{fg=dark-gray}

\setbeamertemplate{navigation symbols}{}
%\setbeamertemplate{mini frames}[default]
%\setbeamercovered{dynamics}
\setbeamerfont*{title}{size=\Large,series=\bfseries}
\setbeamerfont{footnote}{size=\tiny}

%\setbeameroption{notes on second screen} %Dual-Screen Notes
%\setbeameroption{show only notes} %Notes Output

\setbeamertemplate{frametitle}{\vspace{.5em}\bfseries\insertframetitle}
\newcommand{\heading}[1]{\noindent \textbf{#1}\\ \vspace{1em}}

\usepackage{bbding,color,multirow,times,ccaption,tabularx,graphicx,verbatim,booktabs}
\usepackage{colortbl} %Table overlays
\usepackage[english]{babel}
%\usepackage[latin1]{inputenc}
%\usepackage[T1]{fontenc}
\usepackage{lmodern}

%\author[]{Thomas J. Leeper}
\institute[]{
  \inst{}%
  Department of Government\\London School of Economics and Political Science
}

\usepackage{tikz}
\usetikzlibrary{shapes,arrows,decorations.pathreplacing,calc}
\usepackage[normalem]{ulem}

\title{Participant Observation}

\date[]{}

\begin{document}

\frame{\titlepage}

\frame{\tableofcontents}

\section{Interviewing, Continued}
\frame{\tableofcontents[currentsection]}



\frame{
\frametitle{{\normalsize Evaluating a questionnaire}}

\vspace{-1em}

\small

\begin{itemize}\itemsep-0.25em
	\item<1-> Is the question easy for respondents to understand?
	\item<2-> Are the number and types of response options appropriate?
	\item<3-> Are the categories sufficiently distinct from one another?
	\item<4-> Is a ``no opinion,'' ``don't know,'' or ``neither support nor oppose'' response option available?
	\item<5-> Is one survey item (i.e., one question) sufficient to measure this construct?
	\item<6-> How long does it take to read and answer this question?
\end{itemize}
}

\frame{

\frametitle{{\normalsize Cognitive interviewing methods}}

\vspace{-1em}

\small

\begin{itemize}\itemsep-0.25em
\item<1-> Retrospective think-alouds (in which respondents describe how they arrive at their answers either just after they provide them or at the end of the interview)
\item<2-> Paraphrasing (in which respondents restate the question in their own words)
\item<3-> Definitions (in which respondents provide definitions for the key terms in the question)
\item<4-> Probes (in which respondents answer follow-up questions designed to reveal their response strategies)
\end{itemize}

}


\frame{
	\frametitle{Problem Set 5}
	\begin{itemize}\itemsep1em
	\item Any questions about Problem Set 5?
	\end{itemize}
}




\section{Participant Observation}
\frame{\tableofcontents[currentsection]}



\frame{
\frametitle{Focus Groups}
\begin{itemize}\itemsep1em
\item Definition: ``A discussion among a small number of members of a target population, guided by a moderator.''\footnote{Groves et al. 2009. \textit{Survey Methodology}. 2nd Edition. Wiley.}
\item A bridge between interviewing and ethnography
	\begin{itemize}
	\item Less structured than an interview
	\item Typically brief (1-2 hours)
	\item Gather mostly \textit{qualitative} data
	\end{itemize}

% have any of you ever been in a focus group?
\end{itemize}
}


\frame{
\frametitle{``Field work''}
\begin{itemize}\itemsep0.5em
\item Any research activity outside the university setting % soaking and poking
	\begin{itemize}\footnotesize
	\item Textual or archival searches
	\item Interviews (structured or unstructured)
	\item Focus groups
	\item Participant observation
	\item Some mix of these
	\end{itemize}
\item Term is agnostic about approach, theory, and method
\item Might be one-off, sporadic, or long-term
\end{itemize}
}


\frame{
\frametitle{Participant Observation}

\small
\begin{itemize}
\item Definition: ``Participant observation is a research strategy whereby the researcher becomes involved in a social situation for the purpose of understanding the behaviour of those engaged in the setting\dots The outcome of the research is a detailed account of the activities and behaviour of those involved.''\footnote{p.265 from Burnham et al. 2008. \textit{Research Methods in Politics}. 2nd Edition. Palgrave.}
\item<2-> Intentionally subjective/reflective; no belief in possible observational objectivity
\item<3-> Generally inductive in nature
\end{itemize}
}

% A common approach in case study research
% Breakdown of objectivity is this desirable or not?



\frame{
\frametitle{Assorted Examples}

\small

\begin{enumerate}
\item Fenno, R. 1978. \textit{Home Style: U.S. House Members in their Districts}. Pearson.
\item<2-> Goffman, A. 2014. \textit{On the Run: Fugitive Life in an American City}. Chicago.
\item<3-> Cramer, K. 2016. \textit{The Politics of Resentment}. Chicago.
\item<4-> Nielsen, R.K. 2012. \textit{Ground Wars}. Princeton.
\item<5-> Festinger, Riecken, and Schachter. 1956. \textit{When Prophecy Fails}. Harper.
\end{enumerate}

}


\frame{
\frametitle{Operationalization}

\vspace{-2em}

\begin{center}
\tikzstyle{block} = [rectangle, draw, fill=blue!20!white, text width=5em, text centered, rounded corners, minimum height=1em, node distance=7em]
\begin{tikzpicture}[scale=0.5]
\draw<1-> [block] node at (0,0) (concept) {{\Large Concept}};
\draw<1-> [block] node at (-5, -3) (a1) {Attribute};
\draw<1-> [block] node at (0, -3) (a2) {Attribute};
\draw<1-> [block] node at (5, -3) (a3) {Attribute};
\draw<1-> [->, very thick] (concept) -- (a1);
\draw<1-> [->, very thick] (concept) -- (a2);
\draw<1-> [->, very thick] (concept) -- (a3);

\draw<1-> [block, align=center] node at (-11.5, -1.5) (def) {Concept Definition};
\draw<1-> [decorate,very thick, decoration={brace,amplitude=10pt},xshift=-10pt,yshift=0pt]
(-7.5,-3.4) -- (-7.5,0.5);

\draw<1-> [block] node at (-5, -8) (m1) {{\small Measure(s)}};
\draw<1-> [block] node at (0, -8) (m2) {{\small Measure(s)}};
\draw<1-> [block] node at (5, -8) (m3) {{\small Measure(s)}};

\draw<1-> [->, very thick] (a1) -- (m1);
\draw<1-> [->, very thick] (a2) -- (m2);
\draw<1-> [->, very thick] (a3) -- (m3);

\draw<1-> [block, align=center] node at (-11.5, -6.5) (def) {Operation-alization};
\draw<1-> [decorate,very thick, decoration={brace,amplitude=10pt},xshift=-10pt,yshift=0pt]
(-7.5,-8.5) -- (-7.5,-3.6);


\end{tikzpicture}
\end{center}


}


\frame{
\frametitle{Measurement}

\begin{itemize}\itemsep1em
\item<2-> What kind of observations do these authors make?
\item<3-> What is their unit of analysis?
\item<4-> Does participant observation generate DSOs or CPOs?
\end{itemize}

}




\frame{

\frametitle{Four Ideal Types\footnote{Gold, R. 1958. ``Roles in Sociological Field Observation.'' \textit{Social Forces} 36(3): 217--23.}}

\begin{enumerate}\itemsep1em
\item Complete participant
\item Participant as observer
\item Observer as participant
\item Complete observer
\end{enumerate}

}

\frame{
\frametitle{Complete participant}
\begin{itemize}\itemsep1em
\item Participate without disclosing observer/researcher role
\item Essentially covert (``being undercover'')
\item May be useful in sensitive domains
\item Raises ethical concerns
\end{itemize}
}

\frame{
\frametitle{Participant as observer}
\begin{itemize}\itemsep1em
\item Participate, but not fully
\item Retain explicit observer role
\item Negotiate exact role in the situation and access to group members and information
\end{itemize}
}

\frame{
\frametitle{Observer as participant}
\begin{itemize}\itemsep1em
\item Essentially interviews
\item Limited time frames
\item Note: Sometimes seen as indistinguishable from ``complete observer''
\end{itemize}
}

\frame{
\frametitle{Complete observer}
\begin{itemize}\itemsep1em
\item Purely observer role; no participation
\item Still requires negotiated access in many cases, but may not require the same types of consent as participant roles
\item Easier to keep a distance and avoid ``rapport'' with group members
\end{itemize}
}


\frame{
\frametitle{Participant vs. Observer}

\begin{itemize}
\item Not always a choice
	\begin{itemize}\footnotesize
	\item Access might be limited
	\item Ethical obligations
	\end{itemize}
\item<2-> Is situation public vs. private?
\item<3-> How does your presence change the situation?
\item<4-> How does being a participant change your interpretations of events?
\item<5-> How does being an observer change your interpretation of events?
\end{itemize}
}

% Where is Fenno on the continuum between participant and observer?
% Where is Goffman?


\frame{
\frametitle{{\large Ethnography vs. Journalism}}

\begin{center}
\Large What's the difference?
\end{center}

}









\frame{
\frametitle{Activity!}

\begin{itemize}\itemsep1em
\item Choose one of the following contexts:
	\begin{itemize}\footnotesize
	\item Observe Corbyn's shadow cabinet meetings
	\item Observe the core leadership of Britain First
	\end{itemize}
\item Consider:
	\begin{itemize}
	\item What is your research question?
	\item What kinds of observations do you make?
	\item What constraints are placed on what you can do, say, observe, and record?
	\item What perspectives/biases do you bring to the situation?
	\item Are you a participant, observer, or both?
	\end{itemize}
\end{itemize}

}



\section{Preview}
\frame{\tableofcontents[currentsection]}

\frame{

\frametitle{Schedule: Lent Term}

\footnotesize

11 Interviewing, Structured and Unstructured (Jan. 12) \\
12 Participant Observation (Jan. 19) \\
13 Tabulation and Visualization (Jan. 26) \\
14 Sampling and Representativeness (Feb. 2) \\
15 Statistical Inference (Feb. 9) \\
16 Regression Analysis (Feb. 23) \\
17 Matching and Regression
(Mar. 1) \\
18 Experimental Design and Quasi-Experiments
(Mar. 8) \\
19 Ethics and Research Integrity (Mar. 15) \\
20 Conclusion and Synthesis (Mar. 22) \\

}


\frame{

\frametitle{LT Reading Week}

\begin{itemize}\itemsep1em
\item No lecture or class
\item Opportunity for individual revision
\item Structured revision assignment:
	\begin{itemize}
	\item Develop a study guide for the exam
	\item Peer feedback on Moodle: \url{https://moodle.lse.ac.uk/mod/workshop/view.php?id=509034}
	\end{itemize}
\end{itemize}

}

\frame{

\frametitle{Software}

\begin{itemize}\itemsep1em
\item We will use R
\item Download from: \url{https://cran.r-project.org/}
\item Helpful to bring your laptop to lecture
\end{itemize}

}




\appendix
\frame{}

\end{document}
