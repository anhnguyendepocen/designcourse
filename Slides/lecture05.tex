\documentclass[17pt]{beamer} %Makes presentation
%\documentclass[handout]{beamer} %Makes Handouts
\usetheme{Singapore} %Gray with fade at top
\useoutertheme[subsection=false]{miniframes} %Supppress subsection in header
\useinnertheme{rectangles} %Itemize/Enumerate boxes
\usecolortheme{seagull} %Color theme
\usecolortheme{rose} %Inner color theme

\definecolor{light-gray}{gray}{0.75}
\definecolor{dark-gray}{gray}{0.55}
\setbeamercolor{item}{fg=light-gray}
\setbeamercolor{enumerate item}{fg=dark-gray}

\setbeamertemplate{navigation symbols}{}
%\setbeamertemplate{mini frames}[default]
%\setbeamercovered{dynamics}
\setbeamerfont*{title}{size=\Large,series=\bfseries}
\setbeamerfont{footnote}{size=\tiny}

%\setbeameroption{notes on second screen} %Dual-Screen Notes
%\setbeameroption{show only notes} %Notes Output

\setbeamertemplate{frametitle}{\vspace{.5em}\bfseries\insertframetitle}
\newcommand{\heading}[1]{\noindent \textbf{#1}\\ \vspace{1em}}

\usepackage{bbding,color,multirow,times,ccaption,tabularx,graphicx,verbatim,booktabs}
\usepackage{colortbl} %Table overlays
\usepackage[english]{babel}
%\usepackage[latin1]{inputenc}
%\usepackage[T1]{fontenc}
\usepackage{lmodern}

%\author[]{Thomas J. Leeper}
\institute[]{
  \inst{}%
  Department of Government\\London School of Economics and Political Science
}

\usepackage{tikz}
\usetikzlibrary{shapes,arrows}

\title{Building and Testing Political Science Theories}

% How do we create social science theories based on past evidence and novel observation? What roles do induction and deduction play in contemporary political science?

\date[]{}

\begin{document}

\frame{\titlepage}

\frame{\tableofcontents}

\section{Finish Measurement}
\frame{\tableofcontents[currentsection]}


\section{Assessing Measurement Quality}
\frame{\tableofcontents[currentsection]}

\frame{

\frametitle{{\large Assessing Measurement Quality}}

\begin{enumerate}\itemsep1em
\item Conceptual clarity
\item Construct validity
	\begin{itemize}
	\item Convergent validity
	\item Divergent validity
	\end{itemize}
\item Accuracy and precision
\end{enumerate}

}

\frame{
\frametitle{Assessing Measures I}

\begin{itemize}\itemsep1em
\item Conceptual clarity is about knowing what we want to measure
\item Sloppy concepts make for bad measures
	\begin{itemize}
	\item Ambiguity % multiple meanings or multiple labels
	\item Vagueness % concept without a definition
	\end{itemize}
\item<2-> Revise concept definition as needed
\end{itemize}
}


\frame{
\frametitle{Assessing Measures II}

\begin{itemize}\itemsep0.5em
\item Construct validity is the degree to which a variable measures a concept\footnote{Note: Kellstedt and Whitten call this ``content validity''. They use ``construct validity'' to mean whether a measure has predictive validity (i.e., that the measure is related to measures of other concepts that are theorized to be related).}
\item<2-> Construct validity is \textbf{high} if a variable is a measure of the concept we care about
\item<3-> Construct validity is \textbf{low} if a variable is actually a measure of something else
\end{itemize}
}

% sources: bad concept definition; totally inappropriate measures (using income to measure weight); measure becomes the concept (actual income is replaced by self-reported income)

\frame{

\frametitle{{\large Assessing Construct Validity}}

\begin{itemize}\itemsep1em
\item Multiple measures!
\item Look for:
	\begin{itemize}
	\item Convergence (Convergent validity)
	\item Discrimination (Discriminant validity)
	\end{itemize}
\item<2-> For example, the multi-trait, multi-method matrix
\end{itemize}
}

% two (or more) measures of the same concept are highly correlated; scaling
% two (or more) measures of distinct concepts are not correlated
% Measures of distinct concepts may be correlated if they are causally related to one another, so simple correlations do not mean two measures are necessarily of the same concept

% mention predictive validity (what Kellstedt and Whitten call construct validity)


\section{Theory}
\frame{\tableofcontents[currentsection]}


\frame{
\frametitle{{\large Key Points from Last Week}}

\begin{enumerate}\itemsep1em
\item Theory is about concepts
\item Analysis is about measured variables
\item So our task as scientists is to:
	\begin{itemize}
	\item Find observable implications of theory
	\item Draw theoretical implications from measures
	\end{itemize}
\end{enumerate}
}



% theory: tentative conjecture about the causes of some phenomenon of interest\footnote{Kellstedt and Whitten, p.3}


% theories are arguments about how concepts are causally related

% Gerring's criteria: 
% truth (unknowable)
% precision
% generality
% boundedness/scope conditions
% parsimony
% coherent
% relevance

% falsification

% generality and parsimony



% induction (bottom up) vs. deduction (top down; from theory to observation)
% generally, in political science, we are interested in building theories from earlier more acceptable theoretical premises
% if we believe the world operates in some way, we can draw more specific conjectures about features of the world
% and then test those conjectures using evidence

% yet, the result is that when we collect evidence, sometimes we falsify (or fail to find evidence in support of) our theories
% the results is that we must either amend our premises, amend our argument, or set scope conditions on theory

% these theories then become useful because by drawing generalities about the world, we are able to understand particular events
% when those particular events then deviate from expectations, it invites renewed theorizing, amendment of argument, or the implication of scope conditions

% so science is both deductive and inductive, though the way we talk about science is typically deductive


% activity to practice deductive reasoning


% activity to practice inductive reasoning



% there are debates about whether it is okay to make assumptions that we do not test (see, e.g., Kahneman and Tversky)
% there are also debates about whether theories are useful independent of whether they explain the world as it is (see, e.g., Clark and Primo)
% in the end, however, our goal is to "explain" the world...that has to start from a mix of assumption-based theory, theory development expanding on previously validated theory, and novel data collection


% segue to next week: we "test" theories by drawing observable implications of those theories, such that the theories are "observationally inequivalent" (each theory would produce different expectations about how the world would look), and then updating our beliefs about the alternative theories (such that the one more consistent with evidence is more likely to be "true")



\appendix
\frame{}

\end{document}
