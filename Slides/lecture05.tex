\documentclass[17pt]{beamer} %Makes presentation
%\documentclass[handout]{beamer} %Makes Handouts
\usetheme{Singapore} %Gray with fade at top
\useoutertheme[subsection=false]{miniframes} %Supppress subsection in header
\useinnertheme{rectangles} %Itemize/Enumerate boxes
\usecolortheme{seagull} %Color theme
\usecolortheme{rose} %Inner color theme

\definecolor{light-gray}{gray}{0.75}
\definecolor{dark-gray}{gray}{0.55}
\setbeamercolor{item}{fg=light-gray}
\setbeamercolor{enumerate item}{fg=dark-gray}

\setbeamertemplate{navigation symbols}{}
%\setbeamertemplate{mini frames}[default]
%\setbeamercovered{dynamics}
\setbeamerfont*{title}{size=\Large,series=\bfseries}
\setbeamerfont{footnote}{size=\tiny}

%\setbeameroption{notes on second screen} %Dual-Screen Notes
%\setbeameroption{show only notes} %Notes Output

\setbeamertemplate{frametitle}{\vspace{.5em}\bfseries\insertframetitle}
\newcommand{\heading}[1]{\noindent \textbf{#1}\\ \vspace{1em}}

\usepackage{bbding,color,multirow,times,ccaption,tabularx,graphicx,verbatim,booktabs}
\usepackage{colortbl} %Table overlays
\usepackage[english]{babel}
%\usepackage[latin1]{inputenc}
%\usepackage[T1]{fontenc}
\usepackage{lmodern}

%\author[]{Thomas J. Leeper}
\institute[]{
  \inst{}%
  Department of Government\\London School of Economics and Political Science
}


\title{Description and Evidence Gathering}



\date[]{}

\begin{document}

\frame{\titlepage}

\frame{\tableofcontents}

\section{Review Quantitative Description}
\frame{\tableofcontents[currentsection]}

% challenges?

% Data set observation: score for a case on a variable



\section{Observation and Description}
\frame{\tableofcontents[currentsection]}

% cases


% goals of description


% research questions


% types of evidence





\section{Texts as Sources}
\frame{\tableofcontents[currentsection]}


\frame{

\frametitle{What counts as text?}

\begin{itemize}\itemsep1em
\item Tertiary sources
\item Secondary sources
\item Primary sources
\end{itemize}

}

% tertiary: compendia or indices of two other types of sources
% secondary: interpretations of raw evidence
% primary: raw evidence

% brainstorm examples


\frame{

\frametitle{How do you use texts?}

\small

\begin{itemize}\itemsep0.75em
\item Think about your own experience reading, interpreting, and interacting with textual sources for academic purposes (e.g, for writing a term paper).
\item With the person sitting next to you, discuss:
	\begin{enumerate}
	\item The process by which you try to understand the meaning and content of texts
	\item How you choose texts to read
	\end{enumerate}
\end{itemize}

}

\frame{

\frametitle{Use of Texts}

\begin{itemize}\itemsep1em
\item<1-> Text as desription
	\begin{itemize}
	\item Rely on text in lieu of direct observation
	\item Possibly serve as basis of CPO
	\end{itemize}
\item<2-> Text as DSOs
	\begin{itemize}
	\item Treat texts as units
	\end{itemize}
\end{itemize}


}

% texts in lieu of observation
% we cannot see everything ourselves, so we rely on evidence

% texts as CPOs vs. DSOs

\frame{

\frametitle{Challenges of Text}

\begin{enumerate}\itemsep1em
\item Subjectivity
\item Historiography
\item Selection bias
\end{enumerate}

}


% selection bias -> when do we stop looking for evidence?
% -> how do we know that we have all of the evidence?
% -> If we go looking for information about a case as something, do we miss evidence that sees that case as a case of something else?

% define historiography

% texts are subjective

% what is this a case of 

% research questions and text: 
%% at one level, texts can be used to answer any research question
%% there are, however, a lot of interesting research questions that are about text per se

% unit of analysis???
% text, paragraph, sentence
% multiple texts that represent some larger unit (speaker/source, country, party)




\frame{

\frametitle{Preview}

% Problem Set 2

% Reading Week

}


\appendix
\frame{}

\end{document}
