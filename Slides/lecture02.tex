\documentclass[17pt]{beamer} %Makes presentation
%\documentclass[handout]{beamer} %Makes Handouts
\usetheme{Singapore} %Gray with fade at top
\useoutertheme[subsection=false]{miniframes} %Supppress subsection in header
\useinnertheme{rectangles} %Itemize/Enumerate boxes
\usecolortheme{seagull} %Color theme
\usecolortheme{rose} %Inner color theme

\definecolor{light-gray}{gray}{0.75}
\definecolor{dark-gray}{gray}{0.55}
\setbeamercolor{item}{fg=light-gray}
\setbeamercolor{enumerate item}{fg=dark-gray}

\setbeamertemplate{navigation symbols}{}
%\setbeamertemplate{mini frames}[default]
%\setbeamercovered{dynamics}
\setbeamerfont*{title}{size=\Large,series=\bfseries}
\setbeamerfont{footnote}{size=\tiny}

%\setbeameroption{notes on second screen} %Dual-Screen Notes
%\setbeameroption{show only notes} %Notes Output

\setbeamertemplate{frametitle}{\vspace{.5em}\bfseries\insertframetitle}
\newcommand{\heading}[1]{\noindent \textbf{#1}\\ \vspace{1em}}

\usepackage{bbding,color,multirow,times,ccaption,tabularx,graphicx,verbatim,booktabs}
\usepackage{colortbl} %Table overlays
\usepackage[english]{babel}
%\usepackage[latin1]{inputenc}
%\usepackage[T1]{fontenc}
\usepackage{lmodern}

%\author[]{Thomas J. Leeper}
\institute[]{
  \inst{}%
  Department of Government\\London School of Economics and Political Science
}

\usepackage{tikz}
\usetikzlibrary{shapes,arrows}

\title{Causality: Explanation versus Prediction}

% Political science is generally concerned with questions of causality. To do that we need to learn to think counterfactually. How do we know that something causes something else? How do we separate ``correlation'' from ``causation''?

\date[]{}

\begin{document}

\frame{\titlepage}

\frame{\tableofcontents}


\frame{
\frametitle{Who am I?}

\small

\begin{itemize}\itemsep0.5em

\item Thomas Leeper

\item Assistant Professor in Political Behaviour

\item Originally from Minnesota (USA); worked in Denmark for past 2.5 years

\item Interested in public opinion and psychology

\item Office hours:\\
Tuesday 9--11 CON 3.21\\
Sign-up on LSE for You\\
Otherwise, email: \href{mailto:t.leeper@lse.ac.uk}{t.leeper@lse.ac.uk}

\end{itemize}

}



\frame{

\frametitle{Who are you?}

\begin{itemize}\itemsep1em

\item Where are you from?

\item What interests you about government or politics?

\item What do you hope to learn from the course?

\end{itemize}

}



\section{Brief Review of Last Week}
\frame{\tableofcontents[currentsection]}

\frame{\huge\vskip20pt\textbf{What did we learn about last week?}}

% research questions
% what makes questions interesting
% causal versus descriptive questions
% reverse versus forward causal inference


\frame{
\frametitle{Continuing that theme\dots}

By the end of today you should be able to:

\begin{itemize}\itemsep1em
\item Identify what makes for a causal relationship
\item Distinguish causation from correlation/association
\item Begin to analyze research problems using counterfactual thinking
\end{itemize}

}

% Today: focusing on what it means for something to be a ``cause''

% Goal is to develop ways of thinking about how to distinguish causation from association/correlation

% examples of associations (tempurature/pirates; crime/ice cream; )


% where do we look for counterfactuals? backward or forward in time? across similar cases?


\section{Causality}
\frame{\tableofcontents[currentsection]}

\frame{
\frametitle{Write for 1 minute}

\huge\centering\vskip10pt\textbf{What makes something a \textit{cause}?}
}


\frame{
\frametitle{Physical causality}

\begin{itemize}\itemsep1em
\item Action and reaction
\item Features: Observable and deterministic
\item Example:
	\begin{itemize}
	\item Picture a ball resting on top of a hill
	\item What happens if I push the ball?
	\end{itemize}
\item Physical causality is easy to see
\end{itemize}
}

\frame{
\frametitle{Correlation I}
\begin{itemize}\itemsep1em
\item Correlation is the non-independence of two variables for a set of observations
\end{itemize}
}

\frame{
\frametitle{Correlation II}
\begin{itemize}
\item \textit{Observation}: A case or unit (e.g., person, country)
\item \textit{Variable}: A dimension that describes an obseration (e.g., income)
\item \textit{Independence}: Variables are unrelated to one another
	\begin{itemize}
	\item Independent: Height and value on a fair dice roll
	\item Non-independent: Height and weight
	\end{itemize}
\end{itemize}
}


\frame{
\frametitle{Correlation III}

\begin{itemize}\itemsep1em
\item Synonyms: correlation, covariation, relationship, association
	\begin{itemize}
	\item ``Effect'' is frequently used to mean correlation
	\item We'll reserve that term for a \textit{causal effect}
	\end{itemize}
\item Any correlation is a potential cause
	\begin{itemize}
	\item X might cause Y
	\item Y might cause X
	\item X and Y might be caused by Z
	\item X and Y might cause Z
	\item There may be no causal relationship
	\end{itemize}
\end{itemize}
}

\frame<1>[label=mill]{

\frametitle{Mill's methods\footnote{Discussed in Holland}}

\begin{itemize}
\item Agreement
\item \textbf<2>{Difference}
\item Agreement and Difference
\item Residue
\item Concomitant variations
\end{itemize}
}

\frame{
\frametitle{Difference}

If an instance in which the phenomenon under investigation occurs, and an instance in which it does not occur, have every circumstance save one in common, that one occurring only in the former; the circumstance in which alone the two instances differ, is the effect, or cause, or an necessary part of the cause, of the phenomenon.
}

\frame{

\frametitle{Four (or five) principles of causality\footnote{From Kellstedt and Whitten}}

\begin{enumerate}
\item Correlation
\item Nonconfounding
\item Direction (``temporal precedence'')
\item Mechanism
\item (Appropriate level of analysis)
\end{enumerate}
}


\frame{\huge\vskip20pt\textbf{Questions?}}



\frame[label=counterfactuals]{
\frametitle{Counterfactual Thinking}

\begin{itemize}\itemsep1em
\item \textit{Counterfactual}: relating to what has not happened or is not the case
\vspace{0.5em}
\item Causal inference involves inferring \textit{what would have happened} in a counterfactual reality \textit{where the potential cause took on a different value}
\end{itemize}

}

% Has anyone read or seen *A Christmas Carol*?

\frame{
\frametitle{``A Christmas Carol''}

\small 
\begin{itemize}
\item 1843 novel by Charles Dickens
\item Ebenezer Scrooge is shown his own future by the ``Ghost of Christmas Yet to Come''
\item Has the choice to either:
	\begin{itemize}
	\item stay on current path (one counterfactual), or 
	\item change his ways (take a different counterfactual)
	\end{itemize}
\end{itemize}
}

\frame{
\frametitle{Causation}

\begin{itemize}\itemsep1em
\item \textit{Causal effect}: The difference between two ``potential outcomes''
	\begin{itemize}
	\item The outcome that occurs if $X = x_1$
	\item The outcome that occurs if $X = x_2$
	\end{itemize}
\item The causal effect of Scrooge's lifestyle is seen in the differences between two potential futures
\end{itemize}

}

\frame{
\frametitle{Fundamental problem of causal inference}

\Large We can only observe any given unit in one reality!

}

\frame{
\frametitle{Two solutions!\footnote{From Holland}}

\begin{enumerate}\itemsep1em
\item Scientific Solution
	\begin{itemize}
	\item All units are identical
	\item Each can provide a perfect counterfactual
	\item Common in, e.g., agriculture, biology
	\end{itemize}
\item<2-> Statistical Solution
	\begin{itemize}
	\item Units are not identical
	\item Random exposure to a potential cause
	\item Effects measured on average across units
	\item Known as the ``Experimental ideal''
	\end{itemize}
\end{enumerate}

}

\frame{
\frametitle{In Political Science}

\small

\begin{itemize}\itemsep0.5em
\item Causal inference is about searching for appropriate counterfactuals
	\begin{itemize}
	\item<2-> \textit{Causal effect}: Difference in an outcome variable between two counterfactuals
	\item<3-> \textit{Causal inference}: A belief that an event or variable exerts a causal effect on an outcome
	\end{itemize}
\item<4-> Where can we look for counterfactuals?
\end{itemize}

}

\frame{

\frametitle{An Example}

\small

\begin{itemize}\itemsep0.5em
\item For example, if we think smoking might cause lung cancer, how would we know?
\vspace{0.5em}
\item How would we know if smoking caused lung cancer for an individual who smoked?
	\begin{itemize}
	\item What's the relevant counterfactual?
	\end{itemize}
\item How would we know if smoking causes lung cancer on average across many individuals?
	\begin{itemize}
	\item What's the relevant counterfactual?
	\end{itemize}
\end{itemize}

}



\appendix
\frame{}

\frame{\frametitle{Mill's Methods}}

\frame{
\frametitle{Agreement}

If two or more instances of the phenomenon under investigation have only one circumstance in common, the circumstance in which alone all the instances agree, is the cause (or effect) of the given phenomenon.
}

\frame{
\frametitle{Difference}

If an instance in which the phenomenon under investigation occurs, and an instance in which it does not occur, have every circumstance save one in common, that one occurring only in the former; the circumstance in which alone the two instances differ, is the effect, or cause, or an necessary part of the cause, of the phenomenon.
}

\frame{
\frametitle{Agreement and Difference}

If two or more instances in which the phenomenon occurs have only one circumstance in common, while two or more instances in which it does not occur have nothing in common save the absence of that circumstance; the circumstance in which alone the two sets of instances differ, is the effect, or cause, or a necessary part of the cause, of the phenomenon.
}

\frame{
\frametitle{Residue}

Subduct from any phenomenon such part as is known by previous inductions to be the effect of certain antecedents, and the residue of the phenomenon is the effect of the remaining antecedents.
}

\frame{
\frametitle{Concomitant variations}

Whatever phenomenon varies in any manner whenever another phenomenon varies in some particular manner, is either a cause or an effect of that phenomenon, or is connected with it through some fact of causation.
}


\end{document}
