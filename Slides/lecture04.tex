\documentclass[17pt]{beamer} %Makes presentation
%\documentclass[handout]{beamer} %Makes Handouts
\usetheme{Singapore} %Gray with fade at top
\useoutertheme[subsection=false]{miniframes} %Supppress subsection in header
\useinnertheme{rectangles} %Itemize/Enumerate boxes
\usecolortheme{seagull} %Color theme
\usecolortheme{rose} %Inner color theme

\definecolor{light-gray}{gray}{0.75}
\definecolor{dark-gray}{gray}{0.55}
\setbeamercolor{item}{fg=light-gray}
\setbeamercolor{enumerate item}{fg=dark-gray}

\setbeamertemplate{navigation symbols}{}
%\setbeamertemplate{mini frames}[default]
%\setbeamercovered{dynamics}
\setbeamerfont*{title}{size=\Large,series=\bfseries}
\setbeamerfont{footnote}{size=\tiny}

%\setbeameroption{notes on second screen} %Dual-Screen Notes
%\setbeameroption{show only notes} %Notes Output

\setbeamertemplate{frametitle}{\vspace{.5em}\bfseries\insertframetitle}
\newcommand{\heading}[1]{\noindent \textbf{#1}\\ \vspace{1em}}

\usepackage{bbding,color,multirow,times,ccaption,tabularx,graphicx,verbatim,booktabs}
\usepackage{colortbl} %Table overlays
\usepackage[english]{babel}
%\usepackage[latin1]{inputenc}
%\usepackage[T1]{fontenc}
\usepackage{lmodern}

%\author[]{Thomas J. Leeper}
\institute[]{
  \inst{}%
  Department of Government\\London School of Economics and Political Science
}

\usepackage{tikz}
\usetikzlibrary{shapes,arrows}

\title{Measurement: Concepts in Practice}

% To study something, we need to be able to observe and measure it. How do we \textit{operationalize} concepts so that we can study political phenomena? What are challenges of measuring concepts? How do we assign quantitative values to observations?

\date[]{}

\begin{document}

\frame{\titlepage}

\frame{\tableofcontents}


\section{Review}
\frame{\tableofcontents[currentsection]}


Concept definition

Classical approach (minimal, maximal, ordinal)

Family resemblance approach





\section{Measurement}
\frame{\tableofcontents[currentsection]}

Recall definition of variable:

\item \textit{Observation}: A case or unit (e.g., person, country)
\item \textit{Variable}: A dimension that describes an observation (e.g., income)



Operationalization = Creating measures for concepts






Measures follow from concept definitions

Definitions of democracy from Gerring: How do we operationalize these things?



Activity around this



Construct Validity



Threats to Construct Validity (Shadish, Cook, and Campbell Table 3.1 (p.73))

- Bad concept definition

- Mono-operation and mono-method bias: when the operationalization mismeasures the concept or when the operationalization itself becomes part of the concept (e.g., self-reported income versus actual income)

- Experimenter expectancies: does the implmeentation of the experiment actually expose units to things other than the concept of interest (e.g., by encouraging particular types of behavior aside from the behavior of interest)

- Compensatory equalization and rivalry: Knowledge of treatment status actually changes behavior

- Treatment diffusion: Stable unit treatment value assumption from Holland



Quantitative measures versus Qualitative measures

Example: democracy




Reliability

Accuracy





An example

Definition: *Opinion* is a summary evaluation of a particular object

Operationalization?


Agree/disagree

Oppose/support

Degree of favorability

Warm/cool

Positive/negative

Implicit/explicit

How many scale points?



\appendix
\frame{}

\end{document}
